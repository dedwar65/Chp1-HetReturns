Empirical evidence shows that individuals face widely varying returns on their assets. This heterogeneity has important implications for modeling wealth inequality. I incorporate heterogeneity in returns into a standard heterogeneous-agent model, and find the distribution required to match the empirical wealth distribution. Introducing return heterogeneity yields a much better match to the wealth moments measured using Suvery of Consumer Finances (SCF) data than assuming homogeneous returns. In particular, the bottom 60\% of the wealth distribution is matched well enough that the aggregate marginal propensity to consume will be aligned with modest estimates. In addition, it has been noted that the equivalence between a wealth tax and capital income tax fails when individuals no longer earn the same return. I compare a wealth tax and a capital income tax set to raise equal revenue. A capital income tax reduces wealth inequality more by targeting high-return households, but each policy entails different welfare trade-offs across the return distribution. There are a number of final exercises, such as estimating a lognormal distribution of returns across households and interpreting the returns heterogeneity required by the model as arising from banking-sector frictions (e.g., incomplete deposit rate pass-through).