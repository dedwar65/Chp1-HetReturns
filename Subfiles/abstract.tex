Empirical evidence shows that individuals face widely varying returns on their assets. This heterogeneity has important implications for modeling wealth inequality. I incorporate heterogeneity in returns into a standard heterogeneous-agent model, and find the distribution required to match the empirical wealth distribution. In both infinite-horizon and life-cycle versions of the model, introducing even modest return heterogeneity yields a much better match to the top-heavy wealth distribution than assuming homogeneous returns. It also aligns more closely with untargeted patterns such as the distribution of marginal propensities to consume and age-wealth profiles. In addition, we compare a wealth tax and a capital income tax set to raise equal revenue. A capital income tax reduces wealth inequality more by targeting high-return households, but each policy entails different welfare trade-offs across the return distribution. I interpret the heterogeneity in returns required by the model as arising from banking-sector frictions (e.g., incomplete deposit rate pass-through) rather than differences in entrepreneurial talent or financial sophistication.