\onlyinsubfile{\setcounter{section}{7}}
\section{Conclusion}
\notinsubfile{\label{sec:Conclusion}}

\par I find the ex-ante heterogeneity in rates of return needed to match wealth moments, which is a common practice in heterogeneous agent macroeconomics. The model sits well with other deviations from the representative agent framework in that it does a good job of producing a simulated distribution of wealth with significant skewness when compared to its empirical counterpart.

\par This paper is slightly different from the literature which intersects HA models and evidence of a persistent component to returns in that I focus less on common explanations for the latter (like entrepreneurial talent and financial sophistication). Instead, I focus more on heterogeneity in the banking sector regarding offered deposit rates. I incorporate that literature in the standard HA framework with a simple, but realistic story in that many households may be \say{stuck} with the bank in or around their neighborhood. That bank has complex financial decisions to make, which ultimately trickles down to the household through the channel of verying deposit rates offered.

\par Although I leave out the possibility of households switching to one bank or another, this story has a similar essence to the financial literacy story when attempting to explain how returns may be heterogeneous across individuals. However, it leaves out the risk associated with portfolio choice. This is a nice feature, since (i) untangling how much of the persistent component of returns comes from risk preferences and from financial sophistication is not so straightforward and (ii) there is significant heterogeneity in returns even when individuals hold no risky assets.

\par As an aside, this model is still a partial equilibrium analysis. The market interest rate is being taken as given. It is not determined by some market clearing condition.

\par I view my model as the simplest implementation of a potential source of heterogeneity. With that in mind, in the simulation of the model and the resulting SMM estimation, I do not add banks as an agent type. Thus, the bank is not responding in every period to the level of deposits they receive after they set the optimal deposit rate based on the demand for foreign deposits that they face. This also means that I avoid choosing a particular scheme of allocating agents in the model to a particular bank. In this way, there are 7 types of banks just as there are 7 types of returns that an agent may receive. 

% \subsubsection{Equilibrium deposit rates and bank owners}

% \par After proposing a profit-maximization problem for banks, a natural question is: where do the profits of the banks go? 

% \par The answer can be seen in the following scenario. Consider the case where there are only two bank, one globally integrated and one local. The globally integrated bank has an elasticity of deposit rates which is very elastic, and the local bank has an elastcity which is extremely inelastic. This suggests that the former will need to offer a deposit rate which is close to the prevailing world interest rate, else they will face a large reduction in their foreign deposit holdings. The latter will offer a deposit rate which is significantly lower than the world interest rate since they do not face this issue.

% \par In this scenario, the globally integrated bank will earn a profit that can be normalized to 1, and the local bank will earn a profit that can be normalized to a value that is strictly greater than 1 in each period. This implies that the stream of revenue for the entire period (or infinity horizon) for the local bank is strictly higher than the stream of revenue associated with the globally integrated bank. Furthermore, from the literature on asset pricing, we know that a setting such as this will result in a share price for owning the local bank which is strictly larger than the share price for owning the globally integrated bank, so that the ratio of revenue stream to share price is equal for the two bank types.

\par

\par