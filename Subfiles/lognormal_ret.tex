\onlyinsubfile{\setcounter{section}{6}}
\section{Extension: Estimating a Lognormal Distribution of Returns}
\notinsubfile{\label{sec:lognorm}}

\par Using Norwegian population data, \cite{aflgdmlp20} document a persistent component to individual returns, illustrated by a nonuniform distribution of estimated fixed effects in the return to net worth for individuals. Motivated by this finding, I rerun the estimation of my model under the assumption that returns are instead lognormally distributed across households. The main findings of the paper are robust to this modification. 

\subsubsection{Comparing the simulated wealth distributions}

\par Since the model without heterogeneous returns is the same regardless of the distributional assumptions, in Figure 8 I show the results for the infinite horizon and life cycle models with heterogeneous returns, lognormally distributed across households. 

\begin{figure}[h]
    \centering
    \begin{minipage}{0.48\textwidth}
        \centering
        \includegraphics[width=\textwidth]{../Figures/Lognorm_PYrrDistNetWorth_2004Plot.png}
    \end{minipage}
    \hfill
    \begin{minipage}{0.48\textwidth}
        \centering
        \includegraphics[width=\textwidth]{../Figures/Lognorm_LCrrDistNetWorth_2004Plot.png}
    \end{minipage}
    \caption{Comparison of PY (left) vs LC (right) R-Dist Models.}
    \label{fig:PYLCLognorm} 
\end{figure}

\par It is clear from the figure that the results from the infinite horizon setting persist here: the life cycle setting is better at matching the observed wealth distribution than the infinite horizon setting, but they both match wealth moments especially well. Again, I include the wealth moments for each of the models and the data in Table 11.  

\input{Tables/Lognorm_PY_LC_wealth_distribution_compare.tex}

\FloatBarrier

\subsection{Untargeted moments}

\par The aggregate MPC is $25.7\%$ and is nearly identical in the the life-cycle setting at $26\%$. Figure 9 presents a breakdown of average MPCs by wealth deciles for both settings. Again, the MPC's are generally higher in the life-cycle setting due to its superior ability to match lower moments of the wealth distribution. 

\begin{figure}[htbp]
\centering
\includegraphics[width=0.8\textwidth]{Tables/Lognorm_PY_LC_MPC_by_WealthDecile_compare.png}
\caption{Infinite-horizon and Life-cycle marginal propensities to consume.}
\label{fig:LognormPYLCMPCCompare}
\end{figure}


\par For the second set of untargeted moments, I include the cumulative wealth shares by age for this case of an estimated lognormal returns distribution. Again, these moments align more closely with the empirical data than the case with no heterogeneity, particularly for the wealth shares of older cohorts. 

\begin{figure}[h]
\centering
\includegraphics[width=0.8\textwidth]{Tables/Emp_Lorenz_by_age_2004.png}
\caption{Empirical Lorenz Curve Targets from the 2004 SCF.}
\label{fig:EmpLorenzTarLogn}
\end{figure}

\begin{figure}[htbp]
\centering
\includegraphics[width=0.8\textwidth]{Tables/Sim_Lorenz_by_age_Lognorm_LCrrDistNetWorth_2004.png}
\caption{Simulated Untargeted Moments with Heterogeneity (R-dist).}
\label{fig:SimLorenzTarDistLogn}
\end{figure}

\subsection{Tax implications}

\par I also consider the tax implications of the model in this setting. I begin by comparing the effects of each of the tax policies on wealth inequality for both the infinite horizon and the life-cycle cases. The same relationship holds here: each effect of the tax is small (since the policies are revenue-equivalent and only raise $1\%$ of aggregate labor income), but the capital income tax is more effective at reducing wealth inequality more.

\input{../Tables/lognormtaxresultsPY.tex}\unskip
\input{../Tables/lognormtaxresultsLC.tex}\unskip

\par For the welfare effects in this setting, I first estimate the expected welfare gains of starting with the wealth tax and switching to the capital income tax. Again, a newborn who does not know their type would prefer the capital income tax over the wealth tax: the lifetime value of the newborn under the capital income regime is higher. 

\begin{table}[!htbp]
\centering
\caption{Expected Welfare Gains from Tax Reform (Lognormal Returns)}
\label{tab:ce_welfare_tax_lognorm}
\begin{threeparttable}
\begin{tabular}{lcc}
\toprule
& \textbf{Infinite horizon} & \textbf{Life-cycle} \\
\midrule
WT vs CIT       & 0.22\% & 0.18\% \\
WT vs Original  & 0.78\% & 0.36\% \\
CIT vs Original & 0.56\% & 0.18\% \\
\bottomrule
\end{tabular}
\begin{tablenotes}[flushleft]
\footnotesize
\item Notes: Entries are consumption-equivalent (CE) welfare gains, $\Delta$, expressed as percent changes under the assumption that individual returns follow a lognormal distribution across households. 
\end{tablenotes}
\end{threeparttable}
\end{table}


\par These welfare effects can be decomposed by return type for the infinite horizon case and by education-return type in the life-cycle scenario. The pattern of results remains the same: only the highest type in each case prefers the wealth tax over the capital income tax. A notable difference between th distributional assumptions is that the estimated lognormal distribution requires less types with returns less than 1 than the uniform distribution does. Despite this, the welfare results remain robust. 

% Requires: \usepackage{booktabs,threeparttable}
\begin{table}[!htbp]
\centering
\caption{Per-Type Welfare Gain and Baseline Return (WT vs CIT, Lognormal Returns)}
\label{tab:ce_per_type_wt_vs_cit_lognorm}
\begin{threeparttable}
\begin{tabular}{lccccccc}
\toprule
& \textbf{Type 1} & \textbf{Type 2} & \textbf{Type 3} & \textbf{Type 4} & \textbf{Type 5} & \textbf{Type 6} & \textbf{Type 7} \\
\midrule
Baseline $R$ (gross) & 0.976 & 1.001 & 1.014 & 1.026 & 1.038 & 1.053 & 1.079 \\
CE $\Delta$ (WT vs CIT, \%) & 0.267\% & 0.342\% & 0.329\% & 0.317\% & 0.308\% & 0.304\% & -0.326\% \\
\bottomrule
\end{tabular}
\begin{tablenotes}[flushleft]
\footnotesize
\item Notes: CE entries are per-type consumption-equivalent welfare gains (pmv-weighted within type), expressed as percent. Positive values favor the wealth tax over the capital income tax for that return type. Baseline $R$ values are pre-tax gross returns (low $\rightarrow$ high) under the lognormal returns specification.
\end{tablenotes}
\end{threeparttable}
\end{table}


% Requires: \usepackage{booktabs,threeparttable,multirow}
\begin{table}[!htbp]
\centering
\caption{Per-Education Per-Type Welfare Gain and Baseline Return (WT vs CIT, Lognormal Returns)}
\label{tab:ce_per_type_wt_vs_cit_lc_lognorm}
\begin{threeparttable}
\begin{tabular}{llccccccc}
\toprule
\multicolumn{2}{c}{} & \textbf{Type 1} & \textbf{Type 2} & \textbf{Type 3} & \textbf{Type 4} & \textbf{Type 5} & \textbf{Type 6} & \textbf{Type 7} \\
\midrule
\multirow{2}{*}{NoHS}
  & Baseline $R$ (gross)        & 0.9363 & 0.9711 & 0.9910 & 1.0084 & 1.0260 & 1.0471 & 1.0865 \\
  & CE $\Delta$ (WT vs CIT, \%)  & 0.132\% & 0.176\% & 0.225\% & 0.275\% & 0.335\% & 0.292\% & -0.114\% \\
\midrule
\multirow{2}{*}{HS}
  & Baseline $R$ (gross)        & 0.9363 & 0.9711 & 0.9910 & 1.0084 & 1.0260 & 1.0471 & 1.0865 \\
  & CE $\Delta$ (WT vs CIT, \%)  & 0.131\% & 0.177\% & 0.229\% & 0.283\% & 0.332\% & 0.250\% & -0.098\% \\
\midrule
\multirow{2}{*}{College}
  & Baseline $R$ (gross)        & 0.9363 & 0.9711 & 0.9910 & 1.0084 & 1.0260 & 1.0471 & 1.0865 \\
  & CE $\Delta$ (WT vs CIT, \%)  & 0.131\% & 0.177\% & 0.228\% & 0.276\% & 0.311\% & 0.218\% & -0.096\% \\
\bottomrule
\end{tabular}
\begin{tablenotes}[flushleft]
\footnotesize
\item Notes: CE entries are consumption-equivalent welfare gains (pmv-weighted within type), expressed as percent. Positive values favor wealth taxation over capital income taxation for that return type. Baseline $R$ are pre-tax gross returns by type (low $\rightarrow$ high) at $t=0$ under the lognormal returns specification.
\end{tablenotes}
\end{threeparttable}
\end{table}


\subsection{Mechanism for bank heterogeneity}

  \par Table 17 captures the estimated returns distribution that best matches the 2004 SCF data on net worth and the corresponding implied elasticities for both the infinte horizon and life-cycle settings.  

  \input{Tables/lognorm_elasticities.tex} 