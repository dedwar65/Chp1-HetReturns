% Define a few objects that are unique to the BST paper
%\input{\ResourcesDir/owner} % Owned by llorracc (draft) or econ-ark (published) ?

\usepackage{footmisc,datetime2,refcount,xr-hyper,hhline,rotating,makecell,subfiles}
\usepackage{subfiles}


% Allow different actions depending on whether document is being processed as
% subfile or being process as standalone
\providecommand{\onlyinsubfile}{}\renewcommand{\onlyinsubfile}[1]{#1}
\providecommand{\notinsubfile}{}\renewcommand{\notinsubfile}[1]{}

% Commands unique to paper
\provideboolean{CDCEdits}\setboolean{CDCEdits}{true}
\setboolean{CDCEdits}{false}
\setboolean{CDCEdits}{true}
\providecommand{\ifCDC}{\ifthenelse{\boolean{CDCEdits}}}

  %% added packages
  
%Key intro packages
\usepackage[utf8]{inputenc}
\usepackage[backend=bibtex, style=authoryear]{biblatex}
\addbibresource{Chp1.bib}
\usepackage{dirtytalk}
\usepackage{cancel}
\usepackage{graphicx}
\usepackage{wrapfig}
\usepackage{caption}
\usepackage{booktabs}
\usepackage{adjustbox}

%% added packages
\usepackage[T1]{fontenc}
\usepackage{sectsty}
\usepackage{amsmath}
\usepackage{amssymb}
\usepackage{amsfonts}
\usepackage{booktabs}
\usepackage{placeins}
\usepackage[font=small,format=plain,labelfont=bf,textfont=normal,justification=justified,singlelinecheck=false]{caption}
%\usepackage[round, authoryear]{natbib}
\usepackage[super]{nth}
\usepackage{dcolumn}
\usepackage{subcaption}
\usepackage{color} 
\usepackage{afterpage}

%% added commands
\newcommand{\myred}[1]{\textcolor{red}{#1}}
\newcommand{\myblue}[1]{\textcolor{blue}{#1}}
\newcolumntype{d}[1]{D{.}{.}{#1}}


% Terms unique to paper
\newtheorem{tm}{Theorem}
\newtheorem{dfn}{Definition}
\newtheorem{lma}{Lemma}
\newtheorem{assu}{Assumption}
\newtheorem{prop}{Proposition}
\newtheorem{cro}{Corollary}

\newtheorem{example}{Example}
\newtheorem{observation}{Observation}
\newcommand{\exm}{\begin{example}}
\newcommand{\exmm}{\end{example}}
\newcommand{\obs}{\begin{observation}}
\newcommand{\obss}{\end{observation}}
\newcommand{\cor}{\begin{cro}}
\newcommand{\corr}{\end{cro}}
\newtheorem{exa}{Example}
\newcommand{\ex}{\begin{exa}}
\newcommand{\exx}{\end{exa}}
\newtheorem{remak}{Remark}
\newcommand{\brmk}{\begin{remak}}
\newcommand{\ermk}{\end{remak}}
\newcommand{\thm}{\begin{tm}}
\newcommand{\nt}{\noindent}
\newcommand{\thmm}{\end{tm}}
\newcommand{\lm}{\begin{lma}}
\newcommand{\lmm}{\end{lma}}
\newcommand{\ass}{\begin{assu}}
\newcommand{\asss}{\end{assu}}
\newcommand{\df}{\begin{dfn}  }
\newcommand{\dff}{\end{dfn}}
\newcommand{\prp}{\begin{prop}}
\newcommand{\prpp}{\end{prop}}
\newcommand{\bqu}{\sloppy \small \begin{quote}}
\newcommand{\equ}{\end{quote} \sloppy \large}

%%%%%%%%%%%%%%%%%%%%%%%%%%%%
\newtheorem{ax}{Axiom}
\newtheorem{ax.V}{V}
\newtheorem{ax.AA}{AA}
%%%%%%%%%%%%%%%%%%%%%%%%%%%

\newcommand{\eq}{\begin{equation}}
\newcommand{\eqq}{\end{equation}}
\newtheorem{claim}{\it Claim}
\newcommand{\cl}{\begin{claim}}
\newcommand{\cll}{\end{claim}}
\newcommand{\bit}{\begin{itemize}}
\newcommand{\eit}{\end{itemize}}
\newcommand{\ben}{\begin{enumerate}}
\newcommand{\een}{\end{enumerate}}
\newcommand{\bcen}{\begin{center}}
\newcommand{\ecen}{\end{center}}
\newcommand{\fn}{\footnote}
\newcommand{\ds}{\begin{description}}
\newcommand{\dss}{\end{description}}
\newcommand{\prf}{\begin{proof}}
\newcommand{\prff}{\end{proof}}
\newcommand{\cs}{\begin{cases}}
\newcommand{\css}{\end{cases}}
\newcommand{\ml}{\item}
\newcommand{\lb}{\label}
\newcommand{\ra}{\rightarrow}
\newcommand{\tra}{\twoheadrightarrow}
\newcommand*{\supp}{\operatornamewithlimits{sup}\limits}
\newcommand*{\inff}{\operatornamewithlimits{inf}\limits}
\newcommand{\nf}{\normalfont}
\renewcommand{\Re}{\mathbb{R}}
\newcommand*{\mmax}{\operatornamewithlimits{max}\limits}
\newcommand*{\mmin}{\operatornamewithlimits{min}\limits}
%\newcommand*{\argmax}{\operatornamewithlimits{arg max}\limits}
%\newcommand*{\argmin}{\operatornamewithlimits{arg min}\limits}
\newcommand{\uhr}{\!\! \upharpoonright  \!\! }

\newcommand{\CR}{\mathcal R}
%\newcommand{L}{\mathcal F}
\newcommand{\CC}{\mathcal C}
\newcommand{\CT}{\mathcal T}
\newcommand{\CS}{\mathcal S}
\newcommand{\CM}{\mathcal M}
\newcommand{\CL}{\mathcal L}
\newcommand{\CP}{\mathcal P}
\newcommand{\CN}{\mathcal N}

\newtheorem{innercustomthm}{Theorem}
\newenvironment{customthm}[1]
  {\renewcommand\theinnercustomthm{#1}\innercustomthm}
  {\endinnercustomthm}
\newtheorem{einnercustomthm}{Extended Theorem}
\newenvironment{ecustomthm}[1]
  {\renewcommand\theeinnercustomthm{#1}\einnercustomthm}
  {\endeinnercustomthm}
  
  \newtheorem{innercustomcor}{Corollary}
\newenvironment{customcor}[1]
  {\renewcommand\theinnercustomcor{#1}\innercustomcor}
  {\endinnercustomcor}
\newtheorem{einnercustomcor}{Extended Theorem}
\newenvironment{ecustomcor}[1]
  {\renewcommand\theeinnercustomcor{#1}\einnercustomcor}
  {\endeinnercustomcor}
    \newtheorem{innercustomlm}{Lemma}
\newenvironment{customlm}[1]
  {\renewcommand\theinnercustomlm{#1}\innercustomlm}
  {\endinnercustomlm}
    \newtheorem{innercustomdf}{Definition}
\newenvironment{customdf}[1]
{\renewcommand\theinnercustomdf{#1}\innercustomdf}
{\endinnercustomdf}


\newcommand{\red}{\textcolor{red}}
\newcommand{\green}{\textcolor{green}}
\newcommand{\blue}{\textcolor{blue}}
\newcommand{\purple}{\textcolor{purple}}
\newcommand{\mred}[1]{\color{red}{#1}\color{black}}
\newcommand{\mblue}[1]{\color{blue}{#1}\color{black}}
\newcommand{\mpurple}[1]{\color{purple}{#1}\color{black}}
\newcommand*{\medcap}{\mathbin{\scalebox{1.5}{\ensuremath{\cap}}}}%
\newcommand*{\medcup}{\mathbin{\scalebox{1.5}{\ensuremath{\cup}}}}%
%\renewcommand{\thefootnote}{\fnsymbol{footnote}}
