
\section{Results}\notinsubfile{\label{sec:results}}
%\setcounter{page}{0}\pagenumbering{arabic}

\subsection{Matching observed inequality in the distribtution of wealth}

\subsubsection{Adding ex-ante heterogeneity in time preferences}

\par In \cite{cstw2017}'s baseline model, heterogeneity in the time preference factor is not accounted for. However, to address this, the model is extended to include this factor. This is done by assuming that different types of households have a time preference factor drawn uniformly from the interval $(\grave{\beta} - \nabla, \grave{\beta} + \nabla)$, where $\nabla$ represents the level of dispersion. Afterward, the model is simulated to estimate the values of both $\grave{\beta}$ and $\nabla$ so that the model matches the inequality in the wealth distribution. To achieve this, the following minimization problem is solved:

$$ \{\grave{\beta}, \nabla\} = \text{arg}\min_{\beta, \nabla} \bigg( \sum_{i=20, 40, 60, 80} (w_{i}(\beta, \nabla)-\omega_i )^{2} \bigg)^{\frac{1}{2}} $$

\par subject to the constraint that the aggregate capital-to-output ratio in this model matches that of the perfect foresight setting:

$$ \frac{K}{Y} = \frac{K_{PF}}{Y_{PF}}. $$

\par Note that $w_i$ and $\omega_i$ give the porportion of total aggregate net worth held by the top $i$ percent in the model and in the data, respectively.

\subsubsection{The analogous exercise for ex-ante heterogenous rates of return}

\par The $\beta$-dist model proves to be useful in a setting where there are heterogeneous time preference factors since it captures an unobservable component of a household's decision-making process. While the microeconomics literature has put in considerable effort to estimate this parameter, there is currently no consensus on its value.

\par Recent studies by \cite{aflgdmlp20} and \cite{lblcps18} have not only estimated the rate of return on asset holdings but have also uncovered significant heterogeneity across households. Given this motivation, the revised model assumes the existence of multiple types of agents, each earning a distinct rate of return on their assets. A calibration exercise akin to the one used in the $\beta$-dist model is then performed. This crucial step involves comparing the resulting endogenous distribution from simulating this calibrated model to its empirical counterpart to determine if there is an ex-ante distribution of rates of return that can match the observable inequality in the wealth distribution. If a distribution of returns to asset holdings satisfies this criterion, the final step involves reconciling this model heterogeneity with the observed differences in rates of return found in the aforementioned literature.

