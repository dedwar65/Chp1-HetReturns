\onlyinsubfile{\setcounter{section}{3}}
\section{Results}\notinsubfile{\label{sec:results}}
%\setcounter{page}{0}\pagenumbering{arabic}

\subsection{The infinite horizon setting}
\subsubsection{The model with no returns heterogeneity}

\par To solve and simulate the model, I follow the calibration scheme captured in table \ref{tab:calib1}.
\input{../Tables/calibPY.tex}\unskip

\par The solution of the model with no heterogeneity in returns (the R-point model) is the one which finds the value for the rate of return $R$ which minimizes the distance between the simulated and empirical wealth shares at the 20th, 40th, 60th, and 80th percentiles. The empirical targets are computed using the 2004 SCF data on household wealth. The estimation procedure finds this optimal value to be $R = 1.0602$.

\subsubsection{Incorporating heterogeneous returns}

\par As noted above, recent studies by \cite{aflgdmlp20} and \cite{lblcps18} have not only estimated the rate of return on asset holdings but have also uncovered significant heterogeneity across households. With this in mind, the next estimation (the R-dist model) assumes the existence of multiple types of agents, each earning a distinct rate of return on their assets.

\par I follow closely the procedure outlined by \cite{cstw2017}. Specifically, I assume that different types of households have a time preference factor drawn from a uniform distribution on the interval $(\grave{R} - \nabla, \grave{R} + \nabla)$, where $\nabla$ represents the level of dispersion. Afterward, the model is simulated to estimate the values of both $\grave{R}$ and $\nabla$ so that the model matches the inequality in the wealth distribution. To achieve this, the following minimization problem is solved:

$$ \{\grave{R}, \nabla\} = \text{arg}\min_{R, \nabla} \bigg( \sum_{i=20, 40, 60, 80} (w_{i}(R, \nabla)-\omega_i )^{2} \bigg)^{\frac{1}{2}} $$

\par subject to the constraint that the aggregate capital-to-output ratio in this model matches the calibrated value $5$. This is towards the upper bound on plausibly calibrated values for the capital-to-output ratio, as much of the literature chooses values between $2$ and  $3$.

\par Note that $w_i$ and $\omega_i$ give the porportion of total aggregate net worth held by the top $i$ percent in the model and in the data, respectively.

\par The estimation procedure finds this optimal values of $R = 1.0204$ and $\nabla = 0.06833$. These parameter values pin down the estimated uniform distribution. In the model, I've chosen to discretize that distribution to 7 chosen points. The performance of the estimation of both the R-point and R-dist models, measured by their ability to match the SCF data, is compared in figure \ref{fig:PYUnif}.

\begin{figure}[h]
    \centering
    \begin{minipage}{0.48\textwidth}
        \centering
        \includegraphics[width=\textwidth]{../Figures/Unif_PYrrPointNetWorth_2004Plot.png}
    \end{minipage}
    \hfill
    \begin{minipage}{0.48\textwidth}
        \centering
        \includegraphics[width=\textwidth]{../Figures/Unif_PYrrDistNetWorth_2004Plot.png}
    \end{minipage}
    \caption{Comparison of R-Point and R-Dist Models.}
    \label{fig:PYUnif} 
\end{figure}

\par Additionally, here is a table compare the relevant moments of the empirical and simulated wealth distributions.

\input{Tables/Unif_PY_wealth_distribution_compare}


\subsection{Incorporating life cycle dynamics into the model}

\par More realistic assumptions regarding the age and education level of households can have important implications for the income and mortality process of households. Here, I extend the model to incorporate these life cycle dynamics.

\par Households enter the economy at time $t$ aged 24 years old and are endowed with an education level $e \in \{D,HS,C\}$, and initial permanent income level $\textbf{p}_0$, and a capital stock $k_0$. The life cycle version of household income is given by:

$$ y_t = \xi_t \textbf{p}_t = (1 - \tau) \theta_t \textbf{p}_t, $$

where $\textbf{p}_t = \psi_t \bar{\psi}_{es} \textbf{p}_{t-1}$ and $\bar{\psi}_{es}$ captures the age-education-specific average growth factor. Households that have lived for $s$ periods have permanent shocks drawn from a lognormal distribution with mean $1$ and variance $\sigma^{2}_{\psi s}$ and transitory shocks drawn from a lognormal distribution with mean $\frac{1}{\cancel{\mho}}$ and variance $\sigma^{2}_{\theta s}$ with probability $\cancel{\mho} = (1-\mho)$ and $\mu$ with probability $\mho$.

\par The normalized version of the age-education-specific consumption-saving problem for households is given by

\begin{eqnarray*}
  v_{es}(m_t) &=& \max_{c_t} u(c_t(m_t)) + \beta \cancel{D}_{es} \mathbb{E}_{t}[\psi_{t+1}^{1-\rho}v_{es + 1}(m_{t+1})] \\
  &\text{s.t.}& \\
  a_t &=& m_t - c_t, \\
  k_{t+1} &=& \frac{a_t}{\psi_{t+1}}, \\
  m_{t+1} &=& (1 + r_t)k_{t+1} + \xi_{t+1}, \\
  a_t &\geq& 0.
\end{eqnarray*}

\par The additional parameters necessary to calibrate the life cycle version of the model are given in table \ref{tab:calib2}. The age-education dependent mean income levels come from \cite{Cagetti2003}. The permanent and transitory shock variances come from \cite{Sabelhaus2010}. The age-education dependent mortality rates come from \cite{Brown2007}.

\input{../Tables/calibLC.tex}\unskip

\par The estimation procedure finds this optimal value to be $R = 1.0431$ for the R-point model in this setting. The estimation procedure for the R-dist model in the life cycle setting finds optimal values of $R = 1.002$ and $\nabla =0.0953$. Notice the improved performance of the estimation in matching the data displayed in figure \ref{fig:LCUnif}.

\begin{figure}[h]
    \centering
    \begin{minipage}{0.48\textwidth}
        \centering
        \includegraphics[width=\textwidth]{../Figures/Unif_LCrrPointNetWorth_2004Plot.png}
    \end{minipage}
    \hfill
    \begin{minipage}{0.48\textwidth}
        \centering
        \includegraphics[width=\textwidth]{../Figures/Unif_LCrrDistNetWorth_2004Plot.png}
    \end{minipage}
    \caption{Comparison of R-Point and R-Dist Models in the Life-Cycle Setting.}
    \label{fig:LCUnif} 
  \end{figure}

\par Additionally, here is a table compare the relevant moments of the empirical and simulated wealth distributions.

\input{Tables/Unif_LC_wealth_distribution_compare.tex} 

  
\FloatBarrier
\subsection{Untargeted moments}

\subsubsection{The marginal propensity to consume}

\par Naturally, adding a source of (ex-ante) heterogeneity beyond the addition of labor income uncertainty to the representative agent framework will allow the simulated distribution of wealth to match moments of the empirical wealth distribution particularly well. Although it is useful to see that the model with heterogeneous returns does a good job of matching the given lorenz targets, we need another way to assess the model's performance.

\par It is well-documented in the literature on heterogeneous agent modeling that models with a precautionary savings motive will lead to a dispersion in the marginal propensity to consume (MPC) across households. This is a key feauture in why the HA framework is able to produce estimates of the aggregate MPC which are close to empirical measurements, like those measured in \cite{tjlp14}. \footnote{There the authors estimate an MPC of about 48\% and document signicantly larger MPCs for households with low cash-on-hand.}

\par In the infinite horizon setting, we already see that the model with and without heterogeneity will have different implications for the aggregate MPC and its dispersion across households. Namely, the aggregate MPC when households earn the same return on their savings is $11.3\%$, while it is $27.5\%$ when heterogeneous returns are present. We decompose this by looking at average MPCs by wealth decile in the figure below.

\begin{figure}[htbp]
\centering
\includegraphics[width=0.8\textwidth]{Tables/PY_MPC_by_WealthDecile_compare.png}
\label{fig:PYMPCWealthDecileCompare}
\end{figure}

\par As you can see, MPCs are much higher in the case where households earn different returns. With heterogeneous returns, the model generates the observed extreme wealth concentration. Poor households are not just somewhat poorer—they're dramatically poorer, pushing them into the hand-to-mouth region where consumption tracks income almost one-for-one.

\par Since the life cycle setting generates a more realistic distribution of wealth (even without heterogeneous returns), it is reasonable that MPCs are slightly higher here: an aggregate MPC of $12.6\%$ without heterogeneity and of $27.8\%$ with heterogeneous returns. That said, the stark difference between the model with and without heterogeneity remains present here as well.

\begin{figure}[htbp]
\centering
\includegraphics[width=0.8\textwidth]{Tables/LC_MPC_by_WealthDecile_compare.png}
\label{fig:LCMPCWealthDecileCompare}
\end{figure}


\subsubsection{Wealth shares by age cohort}

\par  I also include age-dependent wealth moments from the same wave of the SCF to serve as another set of untargeted moments~\ref{fig:EmpLorenzTar}. These can be found in the following table.

\begin{figure}[h]
\centering
\includegraphics[width=0.8\textwidth]{Tables/Emp_Lorenz_by_age_2004.png}
\caption{Empirical Lorenz Curve Targets from the 2004 SCF.}
\label{fig:EmpLorenzTar}
\end{figure}

\par The next two tables present the simulated version of the untargeted moments for the model without heterogeneity~\ref{fig:SimLorenzTarPoint}, and then with heterogeneity~\ref{fig:SimLorenzTarDist}. As you can see from the tables below, the age-dependent Lorenz targets that arise from the model again fit the data much better when returns heterogeneity is present versus when it is absent.

\begin{figure}[htbp]
\centering
\includegraphics[width=0.8\textwidth]{Tables/Sim_Lorenz_by_age_Unif_LCrrPointNetWorth_2004.png}
\caption{Simulated Untargeted Moments without Heterogeneity (R-point).}
\label{fig:SimLorenzTarPoint}
\end{figure}

\begin{figure}[htbp]
\centering
\includegraphics[width=0.8\textwidth]{Tables/Sim_Lorenz_by_age_Unif_LCrrDistNetWorth_2004.png}
\caption{Simulated Untargeted Moments with Heterogeneity (R-dist).}
\label{fig:SimLorenzTarDist}
\end{figure}





