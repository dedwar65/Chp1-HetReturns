\onlyinsubfile{\setcounter{section}{4}}
\section{Wealth versus Capital Income Tax}
\notinsubfile{\label{sec:Tax}}

\par We've discussed the role of heterogeneous returns in generating substantial wealth inequality in the simulated model. Heterogeneity in returns also has interesting policy implications.  \cite{Guvenen2023} documents that, when individuals earn different returns on their assets, a capital income tax imposes a larger burden on those more efficient with their capital. In contrast, a wealth tax imposes a larger burden on those who are unproductive with their capital. This redistributive effect of the wealth tax leads to higher aggregate productivity, output, and ultimately larger welfare gains than the capital income tax. Following this reasoning, in this section I consider the quantitative effects of each tax scheme on the distribution of wealth when heterogeneous.

\par I start with an economy in which the distribution of returns matches wealth moments measured in the data. The question then becomes: how does the distribution of wealth change if a revenue equivalent tax rate is applied to every household? A capital income tax will clearly decrease wealth inequality, since only households that earn capital income will see a tax but those with zero or negative returns will see a capital tax of zero. Consequently, less wealthy households will hold a higher share of the aggregate wealth than when the capital tax is not present.

The effects of the wealth tax are less obvious, since all households hold some wealth and thus will be taxed accordingly. In this setting, the wealth tax enters the household budget constraint as follows: 
\[
m_{t+1} \;=\; \big(1 + r_t - \tau_w\big)\,k_{t+1} + \xi_{t+1}.
\] Similarly, for the capital income tax,
\[
m_{t+1} \;=\; \big(1 + (1-\tau_{ci})\,r_t\big)\,k_{t+1} + \xi_{t+1}.
\] I compute aggregate income as GDP in this setting, and then find the wealth tax rate and the capital tax rate that would raise tax revenues equal to 1\% of GDP. I follow this procedure for the estimated distribution of returns both for the infinite horizon and the life-cycle setting.

\subsection{Effects on the wealth distribution}

\par Next, I apply the tax schemes to each households and compare the resulting wealth distributions for both cases before and after each policy is implemented. Table 5 presents the results of applying each of the tax schemes in the infinite horizon \ref{tab:taxPY} version of the model. 
% required in preamble:
% \usepackage{booktabs}
% \usepackage{graphicx}   % for \resizebox
% \usepackage{hyperref}   % for \hypertarget (optional if you already use it)

\hypertarget{taxPY}{}
\begin{table}[H]
  \centering
  \resizebox{0.7\textwidth}{!}{%
    \begin{tabular}{@{} l c c c c @{}}
      \toprule
      % top row: empty left cell + spanning header for the four Lorenz columns
      \multicolumn{1}{c}{} & \multicolumn{4}{c}{Lorenz points} \\
      \midrule
      % second header row: the four percent columns
      Tax scheme & 20\% & 40\% & 60\% & 80\% \\
      \midrule
      % data rows: first column uses shortstack to produce two-line entry where desired
      None & .87\%  & 4.1\%  & 11.1\%  & 25.4\%  \\[6pt]
      \shortstack[l]{Wealth \\$\tau_w = \text{.56\%}$} & 1\%  & 5.1\%  & 14\%  & 31.4\%  \\[6pt]
      \shortstack[l]{Capital income \\$\tau_{ci} = \text{10.3\%}$} & 1.2\%  & 5.7\%  & 15.4\%  & 34.1\%  \\
      \bottomrule
    \end{tabular}%
  }
  \caption{Tax policies in the infinite horizon setting.}
  \label{tab:taxPY}
\end{table}
\unskip

\par Applying the tax schemes increases the share of wealth held by households at each chosen percentile, reflecting the redistributive effect of taxation on extreme wealth. Although the effect is mininal in both cases, the capital income tax more effectively reduces wealth inequality than the flat wealth tax. This result is consistent with the extant literature:  when heterogeneous returns are present, a capital income tax disproportionately burdens households who use capital most productively, thereby reducing the concentration of wealth. 

\par Next, I apply the tax schemes in the life-cycle \ref{tab:taxLC} version of the model and obtain similar results: the capital income tax has a larger effect on reducing wealth inequality than the wealth tax that raises the same level of tax revenue. Table 6 presents the results, which that , the effects in the life-cycle model are less pronounced than in the infinite horizon model. 

% required in preamble:
% \usepackage{booktabs}
% \usepackage{graphicx}   % for \resizebox
% \usepackage{hyperref}   % for \hypertarget (optional if you already use it)

\hypertarget{taxLC}{}
\begin{table}[H]
  \centering
  \resizebox{0.7\textwidth}{!}{%
    \begin{tabular}{@{} l c c c c @{}}
      \toprule
      % top row: empty left cell + spanning header for the four Lorenz columns
      \multicolumn{1}{c}{} & \multicolumn{4}{c}{Lorenz points} \\
      \midrule
      % second header row: the four percent columns
      Tax scheme & 20\% & 40\% & 60\% & 80\% \\
      \midrule
      % data rows: first column uses shortstack to produce two-line entry where desired
      None & .7\%  & 3.2\%  & 9.4\%  & 25.2\%  \\[6pt]
      \shortstack[l]{Wealth \\$\tau_w = \text{.36\%}$} & .73\%  & 3.4\%  & 9.7\%  & 25.6\%  \\[6pt]
      \shortstack[l]{Capital income \\$\tau_{ci} = \text{7.1\%}$} & .79\%  & 3.6\%  & 10.3\%  & 26.6\%  \\
      \bottomrule
    \end{tabular}%
  }
  \capt\unskip

\subsection{Welfare effects of the tax policies}

\par As an additional comparison of the two tax policies when heterogeneous returns are present, I follow \cite{Guvenen2023} and examine newborns' lifetime utility, starting with a returns value drawn from the estimated distribution that best fits the SCF wealth data. I use a consumption-equivalent measure of welfare, computed as the percentage change in consumption that would make a newborn (who does not yet know their return type) indifferent between the wealth tax and the capital income tax. Additionally, I decompose the effects of the policies by determining the consumption-equivalent measure of welfare for each estimated return type.

\par For this analysis, I retain my calibrated value of $\rho =1$ such that $u(c) = \log c$. The consumption-equivalent change $\Delta$ between a wealth tax (WT) and a capital income tax (CIT) is defined by equating the lifetime utilities under each policy. Thus, $$ 1 + \Delta = \exp( \frac{V^{WT} - V^{CIT}}{S}  ),  $$ where $S = \sum_{t=0}^{\infty} (\beta \cdot \cancel{D})^{t}$, and $V^{j}$ is the lifetime value of a newborn under regime $j \in \{WT, CIT\}$. The parameter $\Delta$ represents the \textit{constant percentage change in consumption each period that would make a newborn under the wealth tax regime as well off as under the capital income tax regime}. That is, if $\Delta > 0$, then the newborn would be better off switching to the capital income tax. 
 
\par Table 7 presents the welfare effects of the tax policies for both the infinite horizon and life cycle settings. As a baseline for comparison, the table also includes the expected lifetime utility of the original regime with no tax policy present. 

\begin{table}[!htbp]
\centering
\caption{Expected Welfare Gains from Tax Reform}
\label{tab:ce_welfare_tax}
\begin{threeparttable}
\begin{tabular}{lcc}
\toprule
& \textbf{Infinite horizon} & \textbf{Life-cycle} \\
\midrule
WT vs CIT       & 0.20\% & 0.15\% \\
WT vs No Tax  & 0.85\% & 0.35\% \\
CIT vs No Tax & 0.65\% & 0.20\% \\
\bottomrule
\end{tabular}
\begin{tablenotes}[flushleft]
\footnotesize
\item Notes: WT vs CIT means: the expected welfare gain from switching from the wealth tax to the capital income tax. Entries are consumption-equivalent (CE) welfare gains, $\Delta$, expressed as percent changes. Positive values indicate the row’s left policy yields higher newborn lifetime welfare than the right policy. 
\end{tablenotes}
\end{threeparttable}
\end{table}


\par Results from Table 7 show that the no tax regime is preferred to both tax policies. This is expected, as the model is a partial equilibrium analysis, and therefore does not incorporate the potential benefits of the government's spending of the collected tax revenues. Notably, prior to households knowing their type, newborns would need to be compensated more under the wealth tax than under the capital income tax to be indifferent between either tax policy and the no-tax regime. Moreover, newborns under the wealth tax would also need a $.2\%$ adjustment in income to be indifferent between switching to the capital income tax regime. Although subtle, the partial equilibrium nature of the analysis is important. Notably \cite{Guvenen2023} find that the wealth tax is the preferred policy by their welfare analysis. The key distinction is that my model does not incorporate the channel through which taxes can effect aggregate output, productivity, and consumption. Therefore, the welfare analysis here should be viewed as a measure of the relative cost of each tax regime in terms of expected lifetime utility.

\par Additionally, I decompose these welfare effects by determining the preferred tax regime for each of the possible realized return types in the infinite horizon setting. 

\par Why the expected welfare of a newborn is higher under the capital income tax than the wealth tax (and so the capital income tax is favored). Only the highest return type would have to pay some percentage of their consumption to be indifferent to switching from the wealth tax to the capital income tax. All other agents would require some compensation to reach this point of indifference. While many of the types receive some capital income, it is insufficient for the capital income tax to substantially reduce their expected lifetime utility.

% Requires in preamble:
% \usepackage{booktabs, threeparttable, graphicx}

\begin{table}[!htbp]
\centering
\caption{Per-Type Welfare Gain and Baseline Return (WT vs CIT)}
\label{tab:ce_per_type_wt_vs_cit}
\begin{threeparttable}

\begin{minipage}{\linewidth}
\centering
\resizebox{\linewidth}{!}{%
\begin{tabular}{lccccccc}
\toprule
& \textbf{Type 1} & \textbf{Type 2} & \textbf{Type 3} & \textbf{Type 4} & \textbf{Type 5} & \textbf{Type 6} & \textbf{Type 7} \\
\midrule
Baseline $R$ (gross) & 0.9618 & 0.9813 & 1.0009 & 1.0204 & 1.0399 & 1.0594 & 1.0790 \\
CE $\Delta$ (WT vs CIT, \%) & 0.23\% & 0.27\% & 0.34\% & 0.32\% & 0.30\% & 0.24\% & -0.31\% \\
\bottomrule
\end{tabular}%
} % end resizebox
\end{minipage}

\begin{tablenotes}[flushleft]
\footnotesize
\item Notes: CE entries are consumption-equivalent welfare gains (pmv-weighted within type),\\
  expressed as percent. Positive values favor wealth taxation over capital income taxation \\
  for that return type. Baseline $R$ are pre-tax gross returns by type (low $\rightarrow$ high).
\end{tablenotes}
\end{threeparttable}
\end{table}



\par The life-cycle setting produces a similar phenonmenon and can be seen in Table 9. I decompose the effects further by comparing types within education groups.

% Requires: \usepackage{booktabs, threeparttable, multirow, graphicx}

\begin{table}[!htbp]
\centering
\caption{Per-Education Per-Type Welfare Gain and Baseline Return (WT vs CIT)}
\label{tab:ce_per_type_wt_vs_cit_lc}
\begin{threeparttable}

% Body width locked to \linewidth for threeparttable's measurement:
\begin{minipage}{\linewidth}
\centering
\resizebox{\linewidth}{!}{%
\begin{tabular}{llccccccc}
\toprule
\multicolumn{2}{c}{} & \textbf{Type 1} & \textbf{Type 2} & \textbf{Type 3} & \textbf{Type 4} & \textbf{Type 5} & \textbf{Type 6} & \textbf{Type 7} \\
\midrule
\multirow{2}{*}{NoHS}
  & Baseline $R$ (gross)   & 0.9200 & 0.9472 & 0.9744 & 1.0016 & 1.0289 & 1.0561 & 1.0833 \\
  & CE $\Delta$ (WT vs CIT, \%) & 0.118\% & 0.143\% & 0.182\% & 0.264\% & 0.340\% & 0.135\% & -0.124\% \\
\midrule
\multirow{2}{*}{HS}
  & Baseline $R$ (gross)   & 0.9200 & 0.9472 & 0.9744 & 1.0016 & 1.0289 & 1.0561 & 1.0833 \\
  & CE $\Delta$ (WT vs CIT, \%) & 0.117\% & 0.142\% & 0.184\% & 0.271\% & 0.332\% & 0.115\% & -0.108\% \\
\midrule
\multirow{2}{*}{College}
  & Baseline $R$ (gross)   & 0.9200 & 0.9472 & 0.9744 & 1.0016 & 1.0289 & 1.0561 & 1.0833 \\
  & CE $\Delta$ (WT vs CIT, \%) & 0.116\% & 0.142\% & 0.184\% & 0.267\% & 0.309\% & 0.098\% & -0.106\% \\
\bottomrule
\end{tabular}%
} % end resizebox
\end{minipage}

\begin{tablenotes}[flushleft]
\footnotesize
\item Notes: CE entries are consumption-equivalent welfare gains (pmv-weighted within type),\\
  expressed as percent. Positive values favor wealth taxation over capital income taxation \\
  for that return type. Baseline $R$ are pre-tax gross returns by type (low $\rightarrow$ high) at $t=0$.
\end{tablenotes}
\end{threeparttable}
\end{table}







\par For each row, we see that that the consumption-equivalent welfare gains parameter is positive for six out of seven of the return types. This means that for each education level, these bottom six return types under the wealth tax would require some compensation to be as well off as they would be under the capital income tax. In terms of cost, this makes sense; the capital income tax is less costly for those less productive with a unit of capital.