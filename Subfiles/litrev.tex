\onlyinsubfile{\setcounter{section}{1[<65;64;16M}}
\section{Literature Review}
\notinsubfile{\label{sec:Literature Review}}

% \subsection{Collecting Data on the Distribution of Wealth}

% \par Empirical estimates of the skewness in wealth holdings over time provide valuable insights for this paper. Surveys and the imputation of wealth levels using administrative income tax data (sometimes referred to as the \textit{capitalization method}) are the standard ways of collecting household data on the distribution of wealth for empirical analysis.

% \par \cite{wolff2004} provides an early analysis of measurements of wealth by the Survey of Consumer Finances (SCF)\footnote{ See \cite{kennickell2017a} for an extensive description of the methodology for sampling the wealthist households in the SCF and \cite{kennickell2017b} for an analysis of the performance of the SCF at measuring the wealth of the top 1 percent.} by discussing both the concentration and composition of household wealth 1980s and 1990s. The author's analysis corroborates the story of significant and growing inequality in the distribution of wealth in the U.S. Specifically, although the wealth of the average household grew in the 1990s, most of the gains in wealth and income during this period were enjoyed by the upper 20 percent of the wealth distribution, and especially the top 1 percent. While from 1983 to 2003 the top 1 percent experienced 33 percent of the total growth in net worth (89 percent for the top 20 percent), the average wealth of the poorest 40 percent of households fell by 44 percent during this same time period and had reached roughly \$2,900 by 2001.

% \cite{sz16} employs the capitalization method on tax data from the Internal Revenue Service to estimate the distribution of wealth in the United States for a much longer time period of 1913 to 2012. The usefulness in the authors' approach is that they are able to decompose their measure of wealth and savings into fractiles (i.e. top 1 percent, top 10 percent, bottom 20 percent wealth shares), which allows them to analyze the evolution of wealth over time in a way that is standard in the existing literature on wealth inequality. The authors not only find that inequality in the U.S. wealth distribution is realtively high and has been growing significantly in the later periods of their dataset, but they also attribute this growth primarily to the wealthiest of households. Indeed, they cite that the wealth shares of the top .1 percent of the distribution grew from 7 percent in 1978 to 22 percent in 2012. 

\subsection{Explaining Inequality in the Distribution of Wealth}

\par Although wealth inequality gets a considerable amount of media coverage in modern times, it has been the object of deep reflection by academic and political thinkers for centuries. For this reason, the literature on wealth inequality is rich and has been studied by several disciplines. 

\par \cite{jbab18} review the literature on the documented skewness in the distribution of wealth with historical accounts of the origins of the shape of the wealth distribution. The authors then provide the traditional theoretical explanations of this unequal distribution: (i) skewness in the (exogenous) distribution of earnings, (ii) stochastic returns to wealth and savings, and, importantly, (iii) microfoundations for the evolution of wealth resulting from the consumption and saving behavior of households\footnote{As explored in the next section, the emergence of heterogeneous agent models has been a significant development in investigating this issue. \cite{tb1983}, \cite{ra1994}, and \cite{mh1993} are among the earliest examples.}.
 
\par \cite{De_Nardi2017} survey the literature on the microfoundations of wealth accumulation. A number of possible models of household consumption and saving behavior marked by observable differences between households (beyond demographic differences) lead to greater inequality in wealth accumulation over time. Earnings and rate of return risk, ex-ante heterogeneity in preferences, medical expenses, bequest motives, and entrepreneurship are all potential mechanisms which influence the shape of the distribution of wealth.

\par \cite{Gabaix2016} inroduces the concept of speed of convergence to explain the observed evolution of income inequality over time, particularly within the upper tail of the distribution over the past 40 years. Notably, they show that, replicating the empirical dynamics of inequality requires incorporating two additional forms of heterogeneity in the income process for households than are included in the standard consumption and saving models.\footnote{Note that, although this analysis is about the distribution of income, this literature notably asserts that the distribution of wealth inherits some of its skewness from the distribution of income} The first form is \textit{type dependence} in the income growth rate distribution, which accounts for some households having a higher average income growth rate. Second, \textit{scale dependence} captures the fact that higher income levels are more susceptible to shocks to their income growth. The authors find that the former does a good job of explaining the rapid rise in income inequality, and the latter can generate infinitely fast transitions in inequality.

\subsection{Measurements of heterogeneous rates of return}

\par The rationale behind incorporating heterogeneity in rates of return to asset holdings lies in the use of novel datasets in recent empirical research to quantify the differences in returns among individuals. Using 12 years of Norwegian tax records, \cite{aflgdmlp20} document the heterogeneity in realized returns. These differences occur both across individuals (\textit{type dependence}), within and across asset classes with varying levels of risk, and within wealth deciles, where returns are positively correlated with wealth  (\textit{scale dependence}). Moreover, they find that heterogeneous returns exhibits significant persist over time and are positively correlated across generations. These findings support the assumption of ex-ante heterogeneous rates of return in the buffer-stock savings model of households, and provide a benchmark for comparing the distribution of rates of return to the empirical distribution of wealth, as in \cite{cstw2017}. 

\par \cite{lblcps18} use administrative panel data on the balance sheets of Swedish residents to gauge historical and expected returns, as well as risks associated with asset holdings. Their analysis of portfolio performance supports the finding that heterogeneous returns substantially shape in the levels and growth of top wealth shares over time.

\par \cite{Campbell2019} examine equity holdings in India between 2002 and 2011 and find that heterogeneity in investment returns arising from the inherent randomness associated with risky assets and differences in investment strategies, is a key driver of rising inequality in portfolio holdings during the time period. The authors attribute the scale dependence of equity returns to the tendency of smaller accounts to be less diversified than larger ones.

\par \cite{Deuflhard2018} analyze household savings account investments within a heterogeneous agent, incomplete markets model with a precautionary saving motive and a \textit{single asset} to partially insure against risk. They document substantial type dependence in the rate of return to these safe assets and attribute the heterogeneity in returns to differences in financial sophistication. Notably, explaining differences in investment returns for households is a vital step toward endogenizing this form of ex-ante heterogeneity among households in future models.

\par \cite{altmejd2024} provide causal evidence that financial education leads to significant differences in portfolio returns. Using university application records from the Swedish National Archives and data from the Swedish Income and Wealth registry, they show that indivduals marginally admitted to business or economics programs hold more money in stocks and earn higher raw returns on these holdings than those not admitted.


\subsection{Recent heterogeneous agent models with varying rates of return}

\par Several studies extend the heterogeneous agent framework to allow for households to earn different returns on their assets. Of the models with a single, riskless asset and a partial equilibrium analysis of a distribution of returns across agents, my paper is unqiue in that it generates a realistic mass of agents below the 60-th percentile of the wealth distribution. This is important because, as we will see, this region of the wealth distribution is important for generating a reasonable estimate of the aggregate marginal propensity to consume. That said, I will discuss models that go beyond the modeling choices made in my paper. 

\par \cite{Daminato2024} incorporate heterogeneous returns into the solution of a model of consumption-saving for households. They use data from the PSID to document heterogeneity in returns, which they find is comparable to the returns distribution measured using the Norwegian registry by \cite{aflgdmlp20}.

\par \cite{Benhabib2019} propose an overlapping-generations model that incorporates intergenerational wealth transfers, with agents facing uncertainty regarding both labor and capital income. In an earlier work (\cite{jbabml17}) the authors more explicitly define household preferences for bequests to the next generation. Both studies conclude that the distribution of earnings and differences in rates of savings and bequests are crucial for accurately replicating the tails of the observed wealth distribution.

\par \cite{Guler2022} develop a life-cycle model with endogenous heterogeneity in the rate of return by incorporating households' optimal housing and mortgage decisions. Using this framework, the authors investigate the effects of aggregate fiscal shocks, including one-time stimulus payments and mortgage debt relief programs.

\par \cite{Menzio2025} introduce search frictions in financial markets within a standard infinite-horizon macroeconomic model. A distribution of returns across households arises endogenously in their model, and they use the empirical findings from \cite{aflgdmlp20} regarding the distribution of returns to net worth as notable targeted moments.

\subsection{Estimates of the aggregate MPC}


A substantial empirical literature measures households' marginal propensities to consume (MPCs), consistently finding that the aggregate MPC is far larger---and far more heterogeneous---than implied by representative-agent benchmarks. Across quasi-experimental settings, estimated MPCs typically fall in the range of 0.2 to 0.4. For example, \cite{Johnson2006} show that households spent roughly 20--40 percent of the 2001 tax rebates in the quarter they were received, while \cite{Parker2013} find similar magnitudes for the 2008 stimulus payments. Structural and covariance-based approaches deliver comparable estimates: \cite{Blundell2008} imply quarterly MPCs of about 0.2--0.3, and meta-analytic evidence from \cite{Havranek2020} concludes that most studies cluster in this same range. At the same time, research using administrative or survey data highlights that MPCs can be substantially higher among households with low wealth or low liquidity, often exceeding 0.5 (see \cite{Jappelli2014}; \cite{Fagereng2021}). Taken together, these findings indicate that empirically plausible MPCs span a wide but well-documented interval.

\par The model developed in this paper produces MPCs that fall squarely within these empirically established ranges, even though the MPC distribution is not directly targeted in the calibration. This serves as an additional, independent validation of the framework, demonstrating that a model calibrated primarily to match wealth inequality can nevertheless replicate realistic consumption responses across the wealth distribution. 

\subsection{My contributions}

\par This paper contributes to the literature in at least two ways. First, I  model the labor income uncertainty as a random walk, rather than an AR(1) process. The AR(1) specification implies less uncertainty in household earnings over the life cycle, leading to less accumulation of wealth over the life cycle. By using a the permanent income framework, I account for \textit{as much} of the dispersion of wealth across households as possible with labor income uncertainty. The remaining dispersion in wealth across households that cannot be explained by differences in earnings is attributed to heterogeneity in returns, ultimately leading to modest estimates of differences in returns across households.

\par Second, the life-cycle version of my model is much richer in its calibration of earnings and mortality rates than other studies. Specifically, I use the earnings profile of \cite{Cagetti2003} that distinguishes mean earnings not only by age but by education cohort. I use age-education-dependent mortality rates from \cite{Brown2007}. This approach allows households to be distinguished by both age and education, providing an additional mechanism for explaining the dispersion in wealth holdings. However, the impact of this mechanism is limited, and other sources of ex-ante heterogeneity among households, such as time preferences or the rate of return, are still needed to match wealth moments precisely. Said differently, life-cycle dynamics and labor income uncertainty are not enough to generate a reasonably skewed wealth distribution. 

