\onlyinsubfile{\setcounter{section}{1}}
\section{Literature Review}
\notinsubfile{\label{sec:Literature Review}}

\subsection{Collecting Data on the Distribution of Wealth}

\par First and foremost, as the primary focus of models in this field is the distribution of wealth, empirical estimates of the skewness in wealth holdings over time provide valuable insights for this paper. Surveys and the imputation of wealth levels using administrative income tax data (sometimes referred to as the \textit{capitalization method}) are the standard ways of collecting household data on the distribution of wealth for empirical analysis.

\par \cite{wolff2004} provides an early analysis of measurements of wealth by the Survey of Consumer Finances (SCF)\footnote{ See \cite{kennickell2017a} for an extensive description of the methodology for sampling the wealthist households in the SCF and \cite{kennickell2017b} for an analysis of the performance of the SCF at measuring the wealth of the top 1 percent.} by discussing both the concentration and composition of household wealth 1980s and 1990s. The author's analysis corroborates the story of significant and growing inequality in the distribution of wealth in the U.S. Specifically, although the wealth of the average household grew in the 1990s, most of the gains in wealth and income during this period were enjoyed by the upper 20 percent of the wealth distribution, and especially the top 1 percent. While from 1983 to 2003 the top 1 percent experienced 33 percent of the total growth in net worth (89 percent for the top 20 percent), the average wealth of the poorest 40 percent of households fell by 44 percent during this same time period and had reached roughly \$2,900 by 2001.

\cite{sz16} employs the capitalization method on tax data from the Internal Revenue Service to estimate the distribution of wealth in the United States for a much longer time period of 1913 to 2012. The usefulness in the authors' approach is that they are able to decompose their measure of wealth and savings into fractiles (i.e. top 1 percent, top 10 percent, bottom 20 percent wealth shares), which allows them to analyze the evolution of wealth over time in a way that is standard in the existing literature on wealth inequality. The authors not only find that inequality in the U.S. wealth distribution is realtively high and has been growing significantly in the later periods of their dataset, but they also attribute this growth primarily to the wealthiest of households. Indeed, they cite that the wealth shares of the top .1 percent of the distribution grew from 7 percent in 1978 to 22 percent in 2012. 

\subsection{Explaining Inequality in the Distribution of Wealth}

\par \cite{jbab18} conducted a notable, thorough review of the literature on the documented skewness in the distribution of wealth. The survey begins with historical accounts of the origins of the shape of the wealth distribution, dating back as early to Pareto and Samuelson. The authors then provide the traditional theoretical explanations of this unequal distribution: (i) skewness in the (exogenous) distribution of earnings, (ii) stochastic returns to wealth and savings, and, importantly, (iii) microfoundations for the evolution of wealth resulting from the consumption and saving behavior of hosueholds\footnote{As explored in the next section, the emergence of heterogeneous agent models has been a significant development in investigating this issue. \cite{tb1983}, \cite{ra1994}, and \cite{mh1993} are among the earliest examples.}.

\par \cite{Gabaix2016} define a notion for the speed of convergence to provide an explanation for observed evolution of income inequality over time, specifically in the upper tail of the distribution in the past 40 years in the United States. Notably, the authors show that, in order to match the empirical dynamics of inequality, one needs to allow for more forms of heterogeneity in the income process for households that are not incorporated in the standard consumption and saving models.\footnote{Note that, although this analysis is about the distribution of income, this literature notably asserts that the distirbution of wealth inherits some of its skewness from the distribution of income} The first form is \textit{type dependence} in the income growth rate distribution, which models the case in which some households have a higher average income growth rate. The second form, \textit{scale dependence}, captures the fact that higher income levels are more susceptible to shocks to their income growth. The authors find that former does a good job at explaining this fast rise in income inequality, and the latter can generate infinitely fast transitions in inequality.\footnote{As we will see, these notions of \say{type dependence} and \say{scale dependence} show up in the literature on household heterogeneity and the wealth and income distributions; most importantly, in the discussion on heterogeneous rates of return to wealth.} 

\par \cite{De_Nardi2017} provide another survey of the literature, more focused on the microfoundations for the distribution of wealth. Specifically, the authors note a number of possible extensions of models of household consumption and saving behavior, inspired by observable differences and the demographics of households, which lead to differences in wealth accumulation over time. Earnings and rate of return risk, ex-ante heterogeneity in preferences, medical expenses, bequest motives, and entrepreneurship are all cited as potential avenues to better explain the shape of the distirbution of wealth using the behavior of households.

\subsection{Heterogeneous Agents Macroeconomics: Wealth and the Marginal Propensity to Consume}

\par \cite{Guvenen2011} provide an excellent review of the heterogeneous agent models which have become common frameworks of analysis in the macroeconomics literature in the recent decade. The key insights from this survey can be found in its thorough disucssion on the relationship between the complete markets hypothesis and varying degrees of insurance against risk for households. In addition, the authors make a notable distinction between aggregation and representative-agent models; this is important, although the macroeconomics models with incomplete markets and heterogeneous agents of interest may lack a representative-agent counterpart for which the equilibrium analysis is straighforward\footnote{In fact, as the authors note, some heterogeneous agent models may feature \say{approximate aggregation} properties, where the aggregate implications of the model are very close to the implications of some representative agent model} (since these models can generally be solved analytically), computational and numerical methods may be used to conduct a similar analysis of the equilibrium, aggregate implications of such a model. Lastly, each of these key points of the paper are grounded both in theoretical results regarding static and dynamic economies with optimizing households and firms, and also in empirical evidence on both the inherent risk faced by households and the varying degrees of available insurance against these risks.

\par \cite{Krueger2016} provide another review of the heterogeneous agent macroeconomics literature. Here, more emphasis is placed on empirical evidence of heterogeneity across households (in earnings, income, consumption, and wealth) leading up to and during the Great Recession and incorporating features of the business cycle in the model of household consumption and saving to better match the presented cross-sectional data. Furthermore, the authors present an augemented version of the model incorporating demand externalities to analyze the realtionship between the distribution of wealth and the dynamics of aggregate output.

\par The key insight from the heterogeneous agent macroeconomics literature is in the ability of these models to produce an aggregate marginal propensity to consume which is reasonably close to its empirical counterpart. This is a notable failing of the representative-agent framework. \cite{gkgv22} have conducted an extensive analysis of different classes of models in this area, highlighting their strengths and potential drawbacks. Notably, they find that the heterogeneous agent, incomplete markets framework with a single asset generate a marginal propensity to consume that is too low compared to empirical data. While incorporating ex-ante heterogeneity or behavioral preferences can generate a larger MPC, these models tend to suffer from a \say{missing middle} problem - an equilibrium distribution that is overly polarized at the extremes and underestimates the wealth held by middle-income households. As a result, it is worth exploring whether a model that includes one asset, a precautionary savings motive, and ex-ante heterogeneous rates of return also exhibits this shortcoming.

\par  \cite{ks1998} have developed a model that considers both idiosyncratic and aggregate risk in a household's optimization problem. As an additional exercise, they incorporate heterogeneity in time preference to explain the shape of the wealth distribution. More recently, \cite{cstw2017} have updated the model to include income and time preference heterogeneity, with a more realistic income process that accounts for permanent and transitory shocks to household income. The model's analytical framework is flexible enough to accommodate other potential sources of ex-ante heterogeneity, which can be seen in one of its key equilibrium conditions - the \textit{Growth Impatience Condition} (GIC):\footnote{Note that $\beta$ is the time discount factor, $\rho$ is the coefficient of relative risk aversion, $\cancel{D}$ is the survival probability, $\mathbb{E}[\psi^{-1}]$ is the expectation of receiving permanent shock $\psi$, $R_t = \daleth + r_t$, where $\daleth$ is the depreciation rate of capital and $r_t$ is the interest rate, and $\Gamma$ is labor productivity growth.}

\begin{eqnarray*}
\left(\frac{(R \delta)^{1/\rho}\mathbb{E}[\psi^{-1}]\cancel{D}}{\Gamma}\right) & < & 1.
\end{eqnarray*}

\subsection{Measurements of heterogeneous rates of return}

\par The rationale behind incorporating heterogeneity in rates of return to asset holdings lies in the use of novel datasets in recent empirical research to quantify the differences in returns among individuals. \cite{aflgdmlp20} offer an extensive overview of the role that heterogeneous rates of returns play in wealth distribution, as well as conducting their own systematic analysis of return heterogeneity using 12 years of data from Norway's administrative tax records. The authors' findings reveal substantial differences in the average returns to assets for individuals (\textit{type dependence}), that this heterogeneity is found both within and across classes of assets with varying levels of risk, and that returns are positively correlated with wealth  (\textit{scale dependence}). Moreover, they futher demonstrate that this discovery of heterogeneous returns exhibits significant persistence over time and are positively correlated across generations. Each of these findings provide not only motivation for the assumption of ex-ante heterogeneous rates of return in the buffer-stock savings model of households, but also provide a benchmark to compare the distribution of rates of return resulting from the estimation procedure aimed at best matching the empirical distirbution of wealth, as in \cite{cstw2017}. 

\par \cite{lblcps18} provide more evidence towards the assumption of ex-ante heterogeneity in households' rates of return, as they employ administrative panel data on the balance sheets of Swedish residents to gauge historical and expected returns, as well as risks associated with asset holdings. Like previous studies, they also find that heterogeneous returns play a considerable role in the levels and growth of top wealth shares over time. As their analysis is focused on the portfolio performance of wealthy households, their findings offer more support for the idea that scale dependence is an important feature of the observed heterogeneity in the rate of return for households.

\par \cite{Campbell2019} offers a similar conclusion in their analysis by studying wealth held in equity accounts in India between 2002 and 2011. The authors find that heterogeneity in returns to investment, which can be acheived by both the inherent randomness associated with risky investment and differences in the investment strategies of investors, is a main contributor to the increase in inequality of wealth held in equity portfolios during the time period. Here, the authors attribute the scale dependence associated with the returns to equity portfolios to the finding that smaller accounts tend to be poorly diversified relative to their larger account counterparts. 

\par \cite{Deuflhard2018} provide an important analysis for the heterogeneous agent, incomplete markets model with a precautionary saving motive and a \textit{single asset} to partially insure against risk with by studying the performance of households' investments in savings accounts. Not only do they find substantial type dependence in the rate of return to these safe assets, they also attribute the heterogeneity in returns to differences in financial sophistication. As we will see in the next section, providing an explanation for differences in returns to investments for households is a vital step in potentially endogenizing this form of ex-ante heterogeneity among households in future research. Much like the \cite{aflgdmlp20}, comparing the estimated distirbution of rates of return across households to the results of this paper is crucial, especially since the authors' focus is on heterogeneous returns to  safe assets.

\par \cite{altmejd2024} is a recent work which provides causal evidence of financial education leading to significant differences in portfolio returns. Using university application data from the Swedish National Archives and data from the Swedish Income and Wealth registry, they show that indivduals marginally admitted to business or economics programs not only hold more money in stocks but earn a higher raw return on these holdings than their counterparts.


\subsection{Recent HA models with heterogeneous rates of return}

\par The paper \cite{Daminato2024} incorporatres heterogeneous returns into the solution of a model of consumption-saving for households. There, they use data from the PSID to document heterogeneity in returns, which they state is comparable to that found in the Norwegian registry data used by \cite{aflgdmlp20}.

\par \cite{Benhabib2019} proposes an overlapping generations model that incorporates intergenerational wealth transfers. There, agents face uncertainty regarding both labor and capital income. \cite{jbabml17} undertake a similar exercise, where household preferences for bequests to the next generation are more explicitly defined. Both papers conclude that the distribution of earnings and differences in rates of savings and bequests are crucial in matching the characteristics of the observed wealth distribution's tail ends.

\par \cite{Guler2022} develop a life-cycle model that provides a comprehensive description of households' optimal decision-making to endogenize heterogeneity in the rate of return. In addition to the standard consumption-savings behavior, their model also considers optimal choices regarding housing and mortgage decisions. These modeling choices lead to a structural model that better matches the observable size and skewness of the wealth distribution, using realistic features of the household's decision-making process, as a larger number of individuals actively engage with the housing market than with financial markets. Furthermore, these modeling choices enable the authors to investigate the effects of aggregate fiscal shocks, including one-time stimulus payments and mortgage debt relief programs.