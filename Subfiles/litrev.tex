\onlyinsubfile{\setcounter{section}{1}}
\section{Literature Review}
\notinsubfile{\label{sec:Literature Review}}

\subsection{Collecting Data on the Distribution of Wealth}

\par Empirical estimates of the skewness in wealth holdings over time provide valuable insights for this paper. Surveys and the imputation of wealth levels using administrative income tax data (sometimes referred to as the \textit{capitalization method}) are the standard ways of collecting household data on the distribution of wealth for empirical analysis.

\par \cite{wolff2004} provides an early analysis of measurements of wealth by the Survey of Consumer Finances (SCF)\footnote{ See \cite{kennickell2017a} for an extensive description of the methodology for sampling the wealthist households in the SCF and \cite{kennickell2017b} for an analysis of the performance of the SCF at measuring the wealth of the top 1 percent.} by discussing both the concentration and composition of household wealth 1980s and 1990s. The author's analysis corroborates the story of significant and growing inequality in the distribution of wealth in the U.S. Specifically, although the wealth of the average household grew in the 1990s, most of the gains in wealth and income during this period were enjoyed by the upper 20 percent of the wealth distribution, and especially the top 1 percent. While from 1983 to 2003 the top 1 percent experienced 33 percent of the total growth in net worth (89 percent for the top 20 percent), the average wealth of the poorest 40 percent of households fell by 44 percent during this same time period and had reached roughly \$2,900 by 2001.

\cite{sz16} employs the capitalization method on tax data from the Internal Revenue Service to estimate the distribution of wealth in the United States for a much longer time period of 1913 to 2012. The usefulness in the authors' approach is that they are able to decompose their measure of wealth and savings into fractiles (i.e. top 1 percent, top 10 percent, bottom 20 percent wealth shares), which allows them to analyze the evolution of wealth over time in a way that is standard in the existing literature on wealth inequality. The authors not only find that inequality in the U.S. wealth distribution is realtively high and has been growing significantly in the later periods of their dataset, but they also attribute this growth primarily to the wealthiest of households. Indeed, they cite that the wealth shares of the top .1 percent of the distribution grew from 7 percent in 1978 to 22 percent in 2012. 

\subsection{Explaining Inequality in the Distribution of Wealth}

\par \cite{jbab18} provide an insightful review of the literature on the documented skewness in the distribution of wealth. The survey begins with historical accounts of the origins of the shape of the wealth distribution, dating back as early to Pareto and Samuelson. The authors then provide the traditional theoretical explanations of this unequal distribution: (i) skewness in the (exogenous) distribution of earnings, (ii) stochastic returns to wealth and savings, and, importantly, (iii) microfoundations for the evolution of wealth resulting from the consumption and saving behavior of hosueholds\footnote{As explored in the next section, the emergence of heterogeneous agent models has been a significant development in investigating this issue. \cite{tb1983}, \cite{ra1994}, and \cite{mh1993} are among the earliest examples.}.

\par \cite{Gabaix2016} define a notion for the speed of convergence to provide an explanation for observed evolution of income inequality over time, specifically in the upper tail of the distribution in the past 40 years in the United States. Notably, the authors show that, in order to match the empirical dynamics of inequality, one needs to allow for more forms of heterogeneity in the income process for households that are not incorporated in the standard consumption and saving models.\footnote{Note that, although this analysis is about the distribution of income, this literature notably asserts that the distirbution of wealth inherits some of its skewness from the distribution of income} The first form is \textit{type dependence} in the income growth rate distribution, which models the case in which some households have a higher average income growth rate. The second form, \textit{scale dependence}, captures the fact that higher income levels are more susceptible to shocks to their income growth. The authors find that former does a good job at explaining this fast rise in income inequality, and the latter can generate infinitely fast transitions in inequality.\footnote{As we will see, these notions of \say{type dependence} and \say{scale dependence} show up in the literature on household heterogeneity and the wealth and income distributions; most importantly, in the discussion on heterogeneous rates of return to wealth.} 

\par \cite{De_Nardi2017} provide another survey of the literature, more focused on the microfoundations for the distribution of wealth. Specifically, the authors note a number of possible extensions of models of household consumption and saving behavior, inspired by observable differences and the demographics of households, which lead to differences in wealth accumulation over time. Earnings and rate of return risk, ex-ante heterogeneity in preferences, medical expenses, bequest motives, and entrepreneurship are all cited as potential avenues to better explain the shape of the distirbution of wealth using the behavior of households.

\subsection{Measurements of heterogeneous rates of return}

\par The rationale behind incorporating heterogeneity in rates of return to asset holdings lies in the use of novel datasets in recent empirical research to quantify the differences in returns among individuals. \cite{aflgdmlp20} document extensively the heterogeneity in realized returns using 12 years of data from Norway's administrative tax records. The authors' findings reveal substantial differences in the average returns to assets for individuals (\textit{type dependence}), that this heterogeneity is found both within and across classes of assets with varying levels of risk, and that returns are positively correlated with wealth  (\textit{scale dependence}). Moreover, they futher demonstrate that this discovery of heterogeneous returns exhibits significant persistence over time and are positively correlated across generations. Each of these findings provide not only motivation for the assumption of ex-ante heterogeneous rates of return in the buffer-stock savings model of households, but also provide a benchmark to compare the distribution of rates of return resulting from the estimation procedure aimed at best matching the empirical distirbution of wealth, as in \cite{cstw2017}. 

\par \cite{lblcps18} use administrative panel data on the balance sheets of Swedish residents to gauge historical and expected returns, as well as risks associated with asset holdings. Their analysis of portfolio performs supports the finding that heterogeneous returns play a considerable role in the levels and growth of top wealth shares over time.

\par \cite{Campbell2019} consider wealth held in equity accounts in India between 2002 and 2011 and find that heterogeneity in returns to investment, which can be acheived by both the inherent randomness associated with risky investment and differences in the investment strategies of investors, is a main contributor to the increase in inequality of wealth held in equity portfolios during the time period. Here, the authors attribute the scale dependence associated with the returns to equity portfolios to the finding that smaller accounts tend to be poorly diversified relative to their larger account counterparts. 

\par \cite{Deuflhard2018} provide an important analysis for the heterogeneous agent, incomplete markets model with a precautionary saving motive and a \textit{single asset} to partially insure against risk with by studying the performance of households' investments in savings accounts. Not only do they find substantial type dependence in the rate of return to these safe assets, they also attribute the heterogeneity in returns to differences in financial sophistication. Notably, providing an explanation for differences in returns to investments for households is a vital step in potentially endogenizing this form of ex-ante heterogeneity among households in future research.

\par \cite{altmejd2024} is a recent work which provides causal evidence of financial education leading to significant differences in portfolio returns. Using university application data from the Swedish National Archives and data from the Swedish Income and Wealth registry, they show that indivduals marginally admitted to business or economics programs not only hold more money in stocks but earn a higher raw return on these holdings than their counterparts.


\subsection{Recent HA models with heterogeneous rates of return}

\par The paper \cite{Daminato2024} incorporates heterogeneous returns into the solution of a model of consumption-saving for households. There, they use data from the PSID to document heterogeneity in returns, which they state is comparable to that found in the Norwegian registry data used by \cite{aflgdmlp20}.

\par \cite{Benhabib2019} proposes an overlapping generations model that incorporates intergenerational wealth transfers. There, agents face uncertainty regarding both labor and capital income. \cite{jbabml17} undertake a similar exercise, where household preferences for bequests to the next generation are more explicitly defined. Both papers conclude that the distribution of earnings and differences in rates of savings and bequests are crucial in matching the characteristics of the observed wealth distribution's tail ends.

\par \cite{Guler2022} develop a life-cycle model that provides a comprehensive description of households' optimal decision-making to endogenize heterogeneity in the rate of return, as they consider optimal choices regarding housing and mortgage decisions. These modeling choices enable the authors to investigate the effects of aggregate fiscal shocks, including one-time stimulus payments and mortgage debt relief programs.

\par \cite{Menzio2025} introduce a model of the financial market with search frictions into the standard macroeconomic setting of an infinite-horizon decision problem for households and firms. A distribution of returns across households arises endogenously in their model, and they use the empirical findings from \cite{aflgdmlp20} regarding the distribution of returns to net worth as notable targeted moments.

\subsection{My contributions}

\par After taking stock of the research focused on returns heterogeneity and the distribution of wealth, my paper is different from the literature in at least two ways. The first notable difference is that, my paper models the uncertainty in labor income as a random walk, as opposed to an AR(1) process. The key implication of this modeling difference is that, the AR(1) specification leads to less uncertainty in earnings over the life cycle from the perspective of a household at the start of the horizon for which they are choosing an optimal consumption path. This will lead to less accumulation of wealth over the life cycle. In this way, the permanent income specification that I am using will attempt to describe \textit{as much} of the dispersion of wealth across households as possible with labor income uncertainty. The remaining dispersion in wealth across households which cannot be explained by differences in earnings will be attributed to returns heterogeneity, ultimately leading to more modest estimates of differences in returns across households.

\par The second way in which my paper is different from the existing literature is that the life cycle version of my model is much richer in its calibration of earnings and mortality rates. Specifically, I use an earnings profile provided by \cite{Cagetti2003}, which distinguishes mean earnings not only by age but by education cohort. Furthermore, I use age-education dependent mortality rates provided by \cite{Brown2007}, which is uncommon in this literature. Essentially, I distinguish between households demographically by more than just age by incorporating education levels. Doing so does provide another avenue to explain dispersion in wealth holdings across households. However, it is a limited role and some further ex-ante heterogeneity among households, such as time preferences or the rate of return, is still needed to match wealth moments precisely. 