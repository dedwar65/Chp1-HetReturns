\onlyinsubfile{\setcounter{section}{2}}
\section{Mechanism for returns heterogeneity}
\notinsubfile{\label{sec:Mechanism}}

\par There have been a number of potential explanations proposed regarding the persistent component of returns heterogeneity. Among the more common ones is the idea that there are some business owners in the economy, and variability among their entrepreneurial talent leads to even more heterogeneity in labor income than the assumptions about labor income uncertainty that the standard HA models use. \cite{Cagetti2006} and \cite{Cagetti2009} are notable exmaples of explicitly modeling entrepreneurial talent as a compoenent of the consumption-saving problem for households and assesing the ability of such models in matching empirical wealth moments.

\par Although differences in entreprenuerial talent modeled as variability in labor market productivity allows for the model to better match wealth moments at the upper tail of the distribution, it is an unsavory explanation for the mechanism I have in mind in this paper. Namely, this modeling choice will result in households (firms) with high levels of wealth (capital) earning lower rates of return, and vice versa for households (firms) with low levels of wealth (capital). As mentioned before, \cite{aflgdmlp20} documents scale dependece regarding wealth heterogeneity: that returns, as well as the idiosyncratic, persistent component of returns, are positively correlated with wealth.

\par Another  explanation from the literature which is closer to, but still not exactly the same as, the mechanism which allow two households with the same level of assets to earn a different return on them is financial literacy or sophistication. \cite{Lusardi2014} offers a survey on models which explicitly allow for households to make a costly decision to build up a stock of financial literacy, in turn allowing them access to an investment technology offering a higher average return. \cite{Lusardi2017} show that allowing for such endogeneous financial accumulation in the standard consumption-saving framework again allows the model to match wealth moments particularly well, this time through this returns channel (as opposed to the labor income channel associated with entrepreneurial talent).

\par The costly acquisition of a stock of human capital (i.e. financial knowledge) led to further heterogeneity in the rate of return. Along this line of reason, I investigate another cause for heterogeneity in the rate of return offered on assets by following literature which documents substantial heterogeneity in the banking sector, specifically on rates offered to depositors.

\subsection{Deposit rates and bank global integration}

\par As mentioned before, \cite{Deuflhard2018} and others have shown that there can be substational heterogeniety in rates of return earned by depositors. The key step in incorporating this finding in the standard consumption-saving framework is to identify potential sources of this heterogeniety in banking.

\par Consider the balance sheet of domestic (U.S. in my example) banks within the financial sector and the problem they face in optimally choosing the rate to offer on deposits. In an open economy, one can imagine that foreign investors may also demand to hold some of their assets in a domestic bank. In such a setting, the elasticity of deposit rates relative to changes in foreign deposits can be clearly studied. Moreover, if there is a prevailing market interest rate, which is available to both foreign and domestic lenders, there is a clear transmissions channel: market rates may rise, causing the level of foreign deposits to fall wherever the new market rate is substantially larger than the current deposit rate being offered. Ultimately, this should lead to a corresponding reduction in the offered deposit rate, although the change may not be one-to-one. This mechanism is precisely the one found empirically in a recent paper by \cite{Ahmed2025}.

\par The strength of this transmission channel can be viewed as a potential source of heterogeneity in the banking sector, as certain market characteristics would make the deposit rates offered by some banks less sensitive to changes in foreign deposits than other banks. For example, \cite{Sarkisyan2021} make a distinction between \textit{local} and \textit{globally integrated} banks, and show that \say{global banks lose much more deposits relative to local banks in response to unexpected changed in the federal funds rate} \footnote{ \cite{d'Avernas2024} show a similar finding regarding heterogeneity in deposit rates, but for largers vs smaller banks}.

\par With this in mind, the next step is to extend the standard HA model describing household consumption-saving decisions by explicitly modeling a banking sector with local and globally integrated banks. In this setting, it will be clear how heterogeneity in returns may arise for households which are otherwise the same.