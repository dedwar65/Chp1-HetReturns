\onlyinsubfile{\setcounter{section}{5}}
\section{Extension: Mechanism for Returns Heterogeneity}
\notinsubfile{\label{sec:Mechanism}}

\par We've discussed the main contributions of this paper. From here on, we will discuss extensions of the model which arise naturally in the discussion of the role of heterogeneous returns in explaining wealth inequality. The first extension is regarding a potential source of returns heterogeneity in the model. 

\par Several potential explanations have been proposed for the persistent component of returns heterogeneity. One commonly cited factor is that some individuals are business owners, and variability in their entrepreneurial talent creates greater heterogeneity in labor income than standard HA models assume. \cite{Cagetti2006} and \cite{Cagetti2009} are notable examples of studies that explicitly model entrepreneurial talent as a component of the consumption-saving problem for households and assess the ability of such models to match empirical wealth moments.

\par Although modeling differences in entreprenuerial talent as variability in labor market productivity improves the model's ability to match wealth moments at the upper tail of the distribution, it does not adequately explain mechanism considered here. This modeling choice will result in households (firms) with high levels of wealth (capital) earning lower rates of return, and vice versa for households (firms) with low levels of wealth (capital). As shown by \cite{aflgdmlp20}, both average returns and the persistent, idiosyncratic components of returns increase with wealth, reflecting clear scale dependence. 

\par Another  explanation, closer to my mechanism but still distinct is financial literacy or sophistication. \cite{Lusardi2014} survey models that explicitly allow households to make costly investments in financial literacy, granting them access to an investment technology offering a higher average return. \cite{Lusardi2017} show that incorporating such endogeneous financial accumulation into the standard consumption-saving framework enables models to match wealth moments particularly well, this time through the returns channel rather than the labor income channel associated with entrepreneurial talent.

\par As I am particularly interested in a setting with a single, safe asset available, I examine another alternative source of heterogeneity in the rate of return by drawing on literature documenting substantial variation in the banking sector, specifically the rates offered to depositors.

\subsection{Deposit rates and the sensitity of deposit levels to changes in the market interest rate}

\par \cite{Deuflhard2018} among others, document substational heterogeniety in rates of return earned by depositors. To incorporate this finding in the standard consumption-saving framework, I must identify potential sources of heterogeniety in the banking sector.

\par I begin by considering the balance sheet of banks  and the problem they face in optimally choosing the rate to offer on deposits. In a simple setting, they accept deposits at an offered deposit rate and then hold reserves within the central banking system, earning the market interest rate. This discrepancy between the offered deposit rate and the market interest rate leaves room for banks to earn a profit.

\par Empirical evidence from the U.S. shows that the level of deposits at a given bank will change due to exogenous changes in the federal funds rate. In fact, \cite{Drechsler2017} propose a clear transmissions channel for monetary policy: changes in the federal funds rate may lead banks to widen the interest spread they charge on deposits, which causes deposits to flee the bank. They find a strong, negative relationship between changes in the federal funds rate and the growth rate of deposits, with a 100-basis-point increase in the federal funds rate leading to higher deposit outflows in bank branches in more concentrated markets relative to those in less concentrated markets.\footnote{Branches in \say{more concentrated markets} operate in local deposit markets where a few banks hold large market shares.}

\par Thus, variation in the strength of this transmission channel across banks can be viewed as a potential source of heterogeneity in the banking sector, as certain market characteristics would make the level of deposits offered by some banks less sensitive to changes in the federal funds rate than other banks. For example, \cite{Sarkisyan2021} make a distinction between \textit{local} and \textit{globally integrated} banks, and show that \say{global banks lose much more deposits relative to local banks in response to unexpected changed in the federal funds rate}. Similarly, \cite{dAvernas2024} document heterogeneity in deposit rates between larger and smaller banks.

\par Given these empirical findings, I extend the standard HA model describing household consumption-saving decisions by explicitly modeling a banking sector. In this setting, banks will each solve a similar profit-maximization problem regarding accepting deposits at an offered deposit rate and holding reserves that earn the market interest rate. The key distinction between banks in the model is how sensitive the level of deposits is to changes in the market interest rate, in line with prior empirical evidence. This analysis will help explain how heterogeneity in returns may arise for households that are otherwise the same.

\subsection{Model of heterogeneous deposit rates}

\par I consider a small, open economy in which banks and households are the optimizing agents. This analysis operates in partial equilibrium as the world interest rate is taken as given. I present a simple framework for determining banks' optimal deposit rates. Assuming a Cobb-Douglas aggregate production function, I derive the marginal product of capital (less depreciation) consistent with the capital-to-output ratio and its empirical counterpart. This effective interest rate serves as the world or \say{market} interest rate and, together with the estimated distribution of returns, can be used to estimate the elasticities of foreign deposits to the deposit rates for each of the banks in the model.  

\subsubsection{Assumptions regarding the banking sector}

\par In this setting, the economy features a continuum of banks, identical in all respects except for the elasticity of the level of deposits to changes in the market interest rate.

\par The model is static: each bank sets a deposit rate at the beginning of the horizon based solely on the market interest rate and its given elasticity, without the ability to expand its depositor base. \footnote{For example, compare a bank in a suburb area of Montana (local) versus a bank near downtown Houston, Texas (globally integrated).}.

\par Households, in turn, do not endogenously choose which banks to do business with but instead are assigned a bank at birth and remain with it until death. In this simplified setting, the banking sector merely replaces the assignment of idiosyncratic rates of returns over the time horizon in the standard model. 

\subsubsection{Decision problem for banks accepting deposits}

\par The sole distinction between banks in this model is the sensitivity of their level of deposits to changes in the market interest rate, which is indexed by $\varepsilon_i$. This heterogeneity in deposit elasticity generates variation in returns in the model. I present the decision problem for banks using a simplified version of the framework found in \cite{Paul2024}.

 \par Let $R^m$ be the market rate of return, $R^d$ be the rate of return offered on deposits by a  bank, and $S(R^d, R^m)$ be the level of deposits held at a given bank.

\par Banks solve
\[
\max (R^m - R^d) \cdot S(R^d, R^m)
\]

\par subject to
\[
S(R^d, R^m) = A \left( \frac{R^d}{R^m} \right)^{\varepsilon}.
\]

\par Importantly, the first-order condition implies that the optimal deposit rate for the $i$-th bank is $$ R_i^d = \frac{\varepsilon_i}{1+\varepsilon_i} R^m.  $$

\par This relationship plays a key role in the model: once I calibrate the model for a particular value of $R^m$, estimating a uniform distribution of returns using the simulated method of moments will imply a corresponding distribution of elasticities (i.e., the distribution  that minimizes the distance between simulated and empirical Lorenz wealth moments). The seven discretized points therefore capture seven different deposit rates offered, resulting in varying elasticities among seven different bank types in the model. From the expression above, banks with higher values of $\varepsilon_i$  must set $R_i^d$ closer to $R^m$.

 \subsection{The implied distribution of bank heterogeneity}

  \par Having estimated the distribution of returns that best matches the observed wealth moments, I can use these results, along with the assumptions regarding the bank's decision problem to back out an implied distribution of $\varepsilon$. This parameter reflects how strongly each bank’s deposit base responds to market interest rate changes. In both this simplified setting and the transmission channel empirically documented by \cite{Drechsler2017}, differences in these sensitivities ultimately leads to differences in the deposit rates offered by banks.  

  \par I begin by assuming that a Cobb-Douglas aggregate production function, such that the marginal product of capital can be written as $\alpha \frac{Y}{K}$. With calibrated values of $\delta = .025$ and $\alpha = .36$, and a capital-to-output ratio $3$, this setting has an effective interest rate of $R^m = 1.095$, which can be used as the market interest rate.

  \par Since the model with heterogeneity (i.e., the R-dist model) has seven estimated points for the uniform distribution, the implied estimated points for $\varepsilon$ can be uniquely determined by the expression $$\varepsilon_i = \frac{R_i^d}{R^m - R_i^d} .$$

  \par Table 10 shows, for the infinite horizon and life-cycle settings, the estimated returns distribution that best matches 2004 SCF data on net worth and the corresponding implied elasticities. 

  \begin{table}[!htbp]
\centering
\caption{Estimated Returns and Implied Elasticities}
\label{tab:returns_uniform}
\begin{tabular}{|c|c|c|c|}
\hline
\multicolumn{2}{|c|}{PY} & \multicolumn{2}{|c|}{LC} \\
\hline
Estimated returns & Implied elasticities & Estimated returns & Implied elasticities \\
\hline
0.962 & 7.222 & 0.920 & 5.256 \\
0.981 & 8.634 & 0.947 & 6.408 \\
1.001 & 10.632 & 0.974 & 8.081 \\
1.020 & 13.676 & 1.002 & 10.728 \\
1.040 & 18.877 & 1.029 & 15.553 \\
1.059 & 29.786 & 1.056 & 27.127 \\
1.079 & 67.239 & 1.083 & 92.489 \\
\hline
\end{tabular}
\end{table}
 

  \subsubsection{Interpreting the implied distribution of elasticities}

    \par Next, I assess the model's performance by comparing its implications to established empirical findings on bank deposit sensitivities to changes in the federal funds rate. The implied distribution of elasticities derived from my estimation method can be directly compared to those empirical estimates to assess the validity of my model. In addition, I include wealth shares by age cohort as a set of untargeted moments to further assess the model's fit.  

\par By defining the deposit function as $S(\cdot) = A \left( \frac{R^d}{R^m} \right)^{\varepsilon}$ the parameter $\varepsilon$ can be clearly interpreted as the elasticity of deposits to changes in the market interest rate. Formally,

\[
-\varepsilon = \frac{\partial \ln S(\cdot)}{\partial \ln R^m}. 
\]

\par The elasticity parameter therefore indicates how a one-percentage-point change in the market interest rate translates into a percentage change in deposits. This formulation enables me to directly compare the implied elasticities following the estimation procedure to the empirical evidence on the transmission channel described by \cite{Drechsler2017} regarding the relationship between the federal funds rate and the level of deposits at banks. For example, \cite{Genay2004} find that a $1\%$ change in the Fed funds rate leads to about a  $3\%$ to  $4\%$ change in the level of deposits, depending on the size of the bank.

\par The returns heterogeneity required to match observed wealth inequality using only safe assets (i.e., bank deposits) produces vastly overstated elasticities for the banking sector. This outcome illustrates a limitation of the model: banks must raise deposit rates to attract depositors when the federal funds rate increases, regardless of their size. Consequently, there will be less variation in the optimal deposit rates offered across the banking sector, and banks that do not offer competitive deposit rates will likely find that their depositors switch to other safe investment technologies like money market funds. Since my model does not account for the number of banks in the economy or returns heterogeneity arising from choices between safe assets, it is not too surprising that deposit rate elasticities are not well matched in this setting. That said, the ability of the model to back out a distribution of elasticities under the given assumptions is still useful. 
