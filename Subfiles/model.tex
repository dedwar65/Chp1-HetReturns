\onlyinsubfile{\setcounter{section}{3}}
\section{Model}
\notinsubfile{\label{sec:Model}}

\par Here I present a small, open economy with banks and households as the optimizing agents in the model. This will be a partial equilibrium analysis since the world interest rate is being taken as given. That said, I present a simple framework for describing the optimal behavior for banks setting deposit rates. By assuming that there is a cobb-douglas aggregate production function, I will find the marginal product of capital (less depreciation) that is consistent with the capital to output ratio from the model which matches its empirical counterpart. This effective interest rate will be considered the world or \say{market} interest rate and will be used along with the estimated distribution of heterogeneous returns to back out esetimated values for elasticites of foreign deposits to the deposit rates for each of the banks in the model.  

\subsection{Banking sector}

\par There are a continuum of banks, deemed either globally integrated or local. Whether or not a bank falls within a given category depends solely on the elasticity of their offered deposit rate to changes in the level of foreign deposits.

\par The model is static in that, I assume that the level of foreign deposits at either bank type is constant across all periods. Additionally, a given bank cannot take actions to increase the number of depositors (foreign or domestic) at their given institution \footnote{For example, compare a bank in a suburb area of Montana (local) versus a bank near downtown Houston, Texas (globally integrated).}.

\par Lastly, in this simple version of the model, I assume that households do not endogenously choose a bank type to do business with. Clearly, this would lead us towards the literature on costly human capital acquisition and financial literacy. Instead, I assume that a bank type, either globally integrated or local, is assigned to a household at birth with some probability. The household is \say{stuck} with this bank assignment until death.

\par This means that I will need to think carefully about how I will allocate households to banks (i.e. by ad-hoc means or using available data) in the calibration and simulation of the model.

\subsubsection{Decision problem for banks accepting foreign deposits}

\par Note that we may view the distinction between globally integrated and local banks as a source of bank heterogeneity, indexed by $\varepsilon_i$. This will be the source of returns heterogeneity in the model. With this in mind, I propose a simple version of a model of monopolistic competition among hetergoeneous banks introduced by \cite{Monti1972} and \cite{Klein1971}.

\par Let $R^m$ be the market rate of return, $R^dd$ be the rate of return offered on deposits, and $S(R^d)$ be the level of foreign deposits held at a given bank. Banks solve the following optimization problem

$$ \max (R^m -R_i^d) S_i(R^d) $$

\par subject to the constraint that foreign deposit demand is given by the function $S(R^d) = A_i (R_i^d)^{\varepsilon_i}$, where $A_i$ is some constant. 

\par Importantly, the first order condition implies that the optimal deposit rate for the $i$-th bank is given by $$ R_i^d = \frac{\varepsilon_i}{1+\varepsilon_i} R^m  $$. This is crucial for the model in that, so long as we calibrate the model for a particular value of $R^m$, estimating a uniform distribution of returns using the simulated method of moments will imply a corresponding distribution of elasticities (the one that minimizes the distance between simulated and lorenz wealth moments.) In this way, the 7 discretized points capturing 7 different deposit rates offered corresponds to 7 different bank types, which can be distinguished as either global or local based on their elasticities since, from the expression above, banks with higher values of $\varepsilon_i$ (i.e. global banks)  must set $R_i^d$ closer to $R^m$. The reverse would be true for local banks.\footnote{IN the conclusion, I discuss in detail two major drawbacks on my simple version of a source for returns heterogeneity in this model, which could be exploted in future work.}

\subsection{Households}

\subsubsection{Defining the stochastic income process}

\par Each household's income $(y_t)$ during a given period depends on three main factors. The first factor is the aggregate wage rate $(W_t)$ that all households in the economy face. The second factor is the permanent income component $(p_t)$, which represents an agent's present discounted value of human wealth. Lastly, the transitory shock component $(\xi_t)$ reflects the potential risks that households may face in receiving their income payment during that period. Thus, household income can be expressed as the following:

$$ y_t = p_t \xi_t W_t . $$

\par The level of permanent income for each household is subject to a stochastic process. In line with \cite{mf1957}'s description of the labor income process, we assume that this process follows a geometric random walk, which can be expressed as:

$$ p_t = p_{t-1} \psi_{t}, $$

\par The white noise permanent shock to income with a mean of one is represented by $\psi_t$, which is a significant component of household income. The probability of receiving income during a given period is determined by the transitory component, which is modeled to reflect the potential risks associated with becoming unemployed. Specifically, if the probability of becoming unemployed is $\mho$, the agent will receive unemployment insurance payments of $\mu > 0$. On the other hand, if the agent is employed, which occurs with a probability of $1 - \mho$, the model allows for tax payments $\tau_t$ to be collected as insurance for periods of unemployment. The transitory component is then represented as:

\begin{equation*}
\xi_t =
    \begin{cases}
        \mu & \text{with probability $\mho$,} \\
        (1-\tau_t) l \theta_t & \text{with probability $1-\mho$,}
    \end{cases}
\end{equation*}

\par where $l$ is the time worked per agent and the parameter $\theta$ captures the white noise component of the transitory shock.


\subsubsection{Decision problem for households}

\par This paragraph presents the baseline version of the household's optimization problem for consumption-savings decisions, assuming no ex-ante heterogeneity. In this case, each household aims to maximize its expected discounted utility of consumption $u(c) = \frac{c^{1-\rho}}{1-\rho}$ by solving the following:

$$ \max \mathbb{E}_t \sum_{n=0}^{\infty}(\cancel{D}\beta)^{n} u(c_{t+n}). $$

\par It's worth noting that the setting described here follows a perpetual youth model of buffer stock savings, similar to the seminal work of \cite{ks1998}. To solve this problem, we use the bellman equation, which means that the sequence of consumption functions $\{c_{t+n}\}^{\infty}_{n=0}$ associated with a household's optimal choice over a lifetime must satisfy\footnote{Here, each of the relevant variables have been normalized by the level of permanent income ($c_t = \frac{C_t}{p_t}$, and so on). This is the standard state-space reduction of the problem for numerical tractibility.} 

\begin{eqnarray*}
  v(m_t) &=& \max_{c_t} u(c_t(m_t)) + \beta \cancel{D} \mathbb{E}_{t}[\psi_{t+1}^{1-\rho}v(m_{t+1})] \\
  &\text{s.t.}& \\
  a_t &=& m_t - c_t(m_t), \\
  k_{t+1} &=& \frac{a_t}{\cancel{D}\psi_{t+1}}, \\
  m_{t+1} &=& (\daleth + r_t)k_{t+1} + \xi_{t+1}, \\
  a_t &\geq& 0.
\end{eqnarray*}



%%%%%%%%%%%%%%%%%%%%%%%%%%%%%%%%%%%%%%%%%%%%%%%%%%%%%%%%%
%\subsubsection{The analogy for rates of return}

%\par If we want to explore how different returns to assets can affect the endogenous wealth distribution, it's important to examine the following decomposition of a household's evolution of market resources over time:

%\begin{enumerate}
 % \item Assets at the end of the period are equal to market resources minus consumption:

%$$ a_t = m_t - c_t.  $$
    
 % \item Next period's capital is determined from this period's assets via

%$$ k_{t+1} = \frac{a_t}{\cancel{D}\psi_t}.  $$

%  \item Finally, the transition from the beginning of period $t+1$ when capital has not yet been used to produce output, to the middle of that period when output has been produced and incorporated into resources but has not yet been consumed is:

%$$ m_{t+1} = (\daleth + r_t)K_{t+1} + \xi_{t+1}.  $$

 % \end{enumerate}

%\par It's worth recalling that in this model, the rate of return to capital is represented as $(\daleth + r_t)$. This rate of return is directly related to the endogenous level of wealth, which is determined by the level of capital $K_{t+1}$. Therefore, if there are differences in the rate of return across households, this will result in further disparities in wealth holdings.
%%%%%%%%%%%%%%%%%%%%%%%%%%%%%%%%%%%%%%%%%%%%%%%%%%%%%%%%%%%