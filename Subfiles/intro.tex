
\section{Introduction}\notinsubfile{\label{sec:intro}}
\setcounter{page}{0}\pagenumbering{arabic}


\par The unequal distribution of wealth is an extensively documented phenomenon in numerous countries. Regrettably, this feature has not only endured over time but also intensified in recent years. This point is stressed in a recent article from the Institute for Policy Studies (IPS), which revealed that in 2018, the total wealth of the poorest half of Americans was eclipsed by the combined wealth of the three wealthiest men in the nation. The term \say{richest} denotes one's standing in Forbes magazine's list of the 400 richest individuals. Additionally, the IPS report notes that the combined wealth of the top five richest men on this list skyrocketed by a staggering 123\% from March 2020 to October 2021\footnote{See Inequality.org articles data November 21, 2022: \say{Wealth Inequality in the United States} and \say{Updates: Billionaire Wealth, U.S. Job Losses and Pandemic Profiteers} (date accessed: March 27, 2023)}. 

\par The unequal distribution of wealth has also been a subject of considerable interest throughout history in various fields. The statistics literature, for instance, focused on linking the distribution of income to the observable skewness in wealth distribution. The economics literature went further by establishing microfoundations for wealth accumulation over the life cycle. To that end, the macroeconomics literature on inequality has seen significant growth, with the distribution of wealth among households offering insight into how the economy as a whole responds to aggregate fiscal shocks. The recent stimulus checks issued during the pandemic serve as a timely example of this phenomenon.

\par The macroeconomics literature has undergone significant changes in recent years, with the widespread adoption of models that abandon the traditional representative agent assumption in their analysis. As this setting will require that in equilibrium all agents hold the same level of wealth, it is not a desirable laboratory in terms of producing model objects, like the distribution of wealth, that can be compared to real world counterparts.

\par The first departure from the representative agent framework incorporates an exogenously determined income process that generates a distribution of income among households. One common approach to incorporating heterogeneity is to adopt \cite{mf1957}'s description of a permanent and transitory component in the income process. To account for business cycle dynamics, one can further assume that individuals face some level of potential unemployment in each period, creating a precautionary savings motive for consumers. Given that such uncertainty cannot be fully insured against, the availability of a riskless asset that partially insures against income risk results in households choosing to hold different levels of market resources optimally.

\par  \cite{ks1998}'s seminal work suggests that models assuming heterogeneity in individual income perform well in matching the aggregate capital stock but poorly in matching the distribution of wealth. The next step is to assume there is some ex-ante heterogeneity among households, leading more households to optimally hold lower levels of wealth. \footnote{ \cite{gkgv22}'s  recent work provides a comprehensive survey of incomplete markets models with heterogeneous agents featuring (i) uninsurable idiosyncratic income risk, (ii) a precautionary savings motive, and (iii) an endogenous wealth distribution.}  \cite{cstw2017} adopt this approach and assume that agents differ in their time preferences, which reflects implicit characteristics of households relevant to their lifetime wealth accumulation. The authors find that this assumption of modest heterogeneity in time preferences is sufficient to match both the shape and skewness of the empirical distribution of wealth.

\par It is worth noting that the time preference factor $(\beta)$ is one of the key parameters that influences an individual's equilibrium target level of market resources, but it is not directly observable. Therefore, in order to estimate $\beta$, one would need to gather data through surveys or other methods that allow for the direct acquisition of information from households. On the other hand, estimating differences in the rate of return to financial assets across households is possible, as this variable \textit{is} directly observable. 

\par In this paper I provide further evidence of the heterogeneous agent framework's ability to match wealth moments by adding a single source of heterogeneity across households beyond the realization of ex-post shocks to their income. I allow for households differ in the return earned on assets first in the infinite horizon setting. I further extend the model to allow for rich life cycle dynamics. From there, I compare the effects of revenue-equivalent wealth and capital income tax policies on wealth inequality and on welfare. Additionally, I interpret the heterogeneity in the returns on safe assets earned by households in the context of the transmissions channel of monetary policy and its variation across the banking sector. As a final step, I redo the entire exercise under the assumption that the returns across households are lognormally distributed.