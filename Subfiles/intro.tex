
\section{Introduction}\notinsubfile{\label{sec:intro}}
\setcounter{page}{0}\pagenumbering{arabic}


\par The unequal distribution of wealth is a well documented phenomenon across many countries. Disparities have not only persisted over time but intensified in recent years. In 2018, the aggregate wealth of the nation’s three richest individuals surpassed that of the entire bottom 50\% of the U.S. wealth distribution. This point is stressed in a recent article based on Forbes magazine rankings (Institute for Policy Studies, November 2022)\footnote{See Inequality.org articles data November 21, 2022: \say{Wealth Inequality in the United States} and \say{Updates: Billionaire Wealth, U.S. Job Losses and Pandemic Profiteers} (date accessed: March 27, 2023)}. 

\par The unequal distribution of wealth has also long been a focus of study across disciplines. The statistics literature, for instance, has linked the distribution of income to the observable skewness in wealth distribution, and work in economics established microfoundations for wealth accumulation over the life cycle. Macroeconomic studies have examined the distribution of wealth among households to understand how the economy as a whole responds to aggregate fiscal shocks, as well as the decomposition of those effects by group (like wealth level). 

\par The macroeconomics literature has undergone significant changes in recent years, with the widespread adoption of models that abandon the traditional representative agent assumption in their analysis. As this setting will require that in equilibrium all agents hold the same level of wealth, it is not a desirable laboratory in terms of producing model objects, like the distribution of wealth, that can be compared to real world counterparts.

\par Researchers began incorporating an exogenously determined income process that generates a distribution of income among households.The baseline form of (ex-post) heterogeneity in the model is the description of a permanent and transitory process for income. To account for business cycle dynamics, one can further assume that individuals face some level of potential unemployment in each period, creating a precautionary savings motive for consumers. Given that such uncertainty cannot be fully insured against, the availability of a riskless asset that partially insures against income risk results in households choosing to hold different levels of market resources optimally.

\par  \cite{ks1998}'s seminal work suggests that models assuming heterogeneity in individual income perform well in matching the aggregate capital stock but poorly in matching the distribution of wealth. The next step is to assume there is some ex-ante heterogeneity among households, leading more households to optimally hold lower levels of wealth. \footnote{ \cite{gkgv22}'s  recent work provides a comprehensive survey of incomplete markets models with heterogeneous agents featuring (i) uninsurable idiosyncratic income risk, (ii) a precautionary savings motive, and (iii) an endogenous wealth distribution.}  \cite{cstw2017} adopt this approach and assume that agents differ in their time preferences, which reflects implicit characteristics of households relevant to their lifetime wealth accumulation. The authors find that this assumption of modest heterogeneity in time preferences is sufficient to match both the shape and skewness of the empirical distribution of wealth.

\par The time preference factor $(\beta)$ is one of the key parameters that influences an individual's equilibrium target level of market resources, but it is not directly observable. Estimating would require one to collect data through surveys or other methods that provide direct information from households. In contrast, differences in the rate of return on financial assets across households is possible, as this variable \textit{is} observable. 

\par In this paper, I provide further evidence of the heterogeneous agent framework's ability to match wealth moments by adding a single source of heterogeneity across households beyond the realization of ex-post shocks to their income. I allow households to differ in the return earned on assets first in the infinite horizon setting. I further extend the model to allow for rich life cycle dynamics. From there, I compare the effects of revenue-equivalent wealth and capital income tax policies on wealth inequality and welfare. Additionally, I interpret the heterogeneity in the returns on safe assets earned by households in the context of the transmissions channel of monetary policy and its variation across the banking sector. As a final step, I rerun the entire exercise under the assumption that the returns across households are lognormally distributed.