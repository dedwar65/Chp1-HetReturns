\begin{table}[!htbp]
\centering
\caption{Expected Welfare Gains from Tax Reform}
\label{tab:ce_welfare_tax}
\begin{threeparttable}
\begin{tabular}{lcc}
\toprule
& \textbf{Infinite horizon} & \textbf{Life-cycle} \\
\midrule
WT vs CIT       & 0.20\% & 0.15\% \\
WT vs Original  & 0.85\% & 0.35\% \\
CIT vs Original & 0.65\% & 0.20\% \\
\bottomrule
\end{tabular}
\begin{tablenotes}[flushleft]
\footnotesize
\item Notes: Policy A vs Policy B means: the expected welfare gain from switching from policy A to policy B. Entries are consumption-equivalent (CE) welfare gains, $\Delta$, expressed as percent changes. Positive values indicate the row’s left policy yields higher newborn lifetime welfare than the right policy. Values use pmv (type-probability) weights. Infinite-horizon refers to the perpetual-youth specification; life-cycle refers to finite lives. The policies are revenue-equivalent. 
\end{tablenotes}
\end{threeparttable}
\end{table}
